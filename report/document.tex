% Created 2023-05-11 Thu 15:26
% Intended LaTeX compiler: pdflatex
\documentclass[a4paper,10pt,compsoc,conference]{IEEEtran}
\usepackage[utf8]{inputenc}
\usepackage[T1]{fontenc}
\usepackage{graphicx}
\usepackage{longtable}
\usepackage{wrapfig}
\usepackage{rotating}
\usepackage[normalem]{ulem}
\usepackage{amsmath}
\usepackage{amssymb}
\usepackage{capt-of}
\usepackage{hyperref}
\PassOptionsToPackage{table,xcdraw}{xcolor}
\usepackage[T1]{fontenc}
\usepackage[utf8]{inputenc}
\usepackage{amsmath}
\usepackage{amssymb}
\usepackage{amsthm}
\usepackage{amsfonts}
\usepackage{graphicx}
\usepackage{colortbl}
\usepackage{enumitem}
\usepackage[ruled, lined, longend, linesnumbered]{algorithm2e}
\usepackage{bm}
\usepackage{listings}
\usepackage[dvipsnames]{xcolor}
\usepackage{palatino}
\usepackage{palatino}
\usepackage{caption}
\usepackage{array}
\usepackage{pgf}
\usepackage{lmodern}
\usepackage{import}
\usepackage{layouts}
\usepackage{supertabular}
%\usepackage{xtab,afterpage}
\usepackage{makecell}
\setcounter{secnumdepth}{3}
\setlength{\partopsep}{1em}
\DeclareMathOperator*{\argmax}{arg\,max}
\DeclareMathOperator*{\argmin}{arg\,min}
\lstset{basicstyle=\ttfamily, keywordstyle=\bfseries, language=Python, float}
\captionsetup[table]{skip=10pt,justification=centering}
\renewcommand{\arraystretch}{1.4}
\setcounter{secnumdepth}{3}
\author{Matteo Franzil, Luis Augusto Dias Knob}
\date{2023-05-11}
\title{On Exploiting gzip's Content-Dependent Compression}
\hypersetup{
 pdfauthor={Matteo Franzil, Luis Augusto Dias Knob},
 pdftitle={On Exploiting gzip's Content-Dependent Compression},
 pdfkeywords={},
 pdfsubject={},
 pdfcreator={Emacs 28.2 (Org mode 9.5.5)}, 
 pdflang={English}}
\usepackage{natbib}
\begin{document}

\maketitle
\setlength{\parskip}{1pt}

\section{Introduction}
\label{sec:orgbf69b7c}

Despite the development of more recent compression techniques like \texttt{bzip2}
and \texttt{xz,} \texttt{gzip} is still a well-liked UNIX compression tool. Gzip is
employed as the default compression technique in the majority of Linux and
UNIX distributions. Although it has a somewhat poor compression ratio when
compared to other utilities, this can be justified by its simplicity and
quick compression and decompression times
\citep{GNUGzip,deutschDEFLATECompressedData1996}. This is also true for some
tools, like \texttt{containerd}, which employs gzip as its standard compression
technique for its image layers
\citep{MakeImageLayer,SupportParallelDecompression}.

In this report, we explore the possibility of exploiting gzip's
algorithm for artificially increasing the decompression time of a file. By
creating files filled with semi-random data generated with various methods,
we show that compression and decompression times may vary significantly
depending on the content of the file and the compression level used. Indeed,
we show that the decompression time of a file can be increased by a factor of
3 in the worst case, when compared to the decompression time of a file
containing English text.

\section{Results}
\label{sec:org27037a0}

\subsection{System setup}
\label{sec:org10faca4}

We run our experiments on a machine with a 4-core Intel Xeon Silver 4112 CPU
@ 2.60GHz, 64GB of RAM, running Ubuntu Server 20.04.2 LTS. We used \texttt{gzip}
version 1.12 on both machines. Tests were run in isolation and with
CPU pinning, in order to reduce the impact of other processes on the results,
and were run 5 times to reduce the impact of noise.

Our tests comprised the following steps:

\begin{enumerate}
\item Generate a file of 100MB in size, with a specific content
\item Compress the file with gzip
\item Decompress the file with gzip
\end{enumerate}

We measured the time taken by each step, the size of the compressed and
uncompressed files, the CPU usage, and the compression ratios. We repeated
this test for each of \texttt{gzip}'s compression levels (1-9). 

\subsection{Popular tools}
\label{sec:orga0e3bd6}

We first started by analyzing the compression and decompression times of
files generated with some popular random data generators. We generated files
with the following tools \citep{strandbergLorem2022}:

\begin{itemize}
\item \texttt{od -{}-format=x /dev/urandom | head -c 1G}
\item \texttt{base64 /dev/urandom | head -c 1G}
\item \texttt{cat /dev/urandom | tr -dc 'a-zA-Z0-9' | head -c 1G}
\item \texttt{openssl rand -out myfile "\$( echo 1G | numfmt -{}-from=iec )"}
\item \texttt{lorem -c 1000000000}
\end{itemize}

We then compressed and decompressed these files with gzip, using the methods
described above. The decompression times are shown in \autoref{fig:popular-tools}.

\begin{figure}
  \begin{center}
    %% Creator: Matplotlib, PGF backend
%%
%% To include the figure in your LaTeX document, write
%%   \input{<filename>.pgf}
%%
%% Make sure the required packages are loaded in your preamble
%%   \usepackage{pgf}
%%
%% Also ensure that all the required font packages are loaded; for instance,
%% the lmodern package is sometimes necessary when using math font.
%%   \usepackage{lmodern}
%%
%% Figures using additional raster images can only be included by \input if
%% they are in the same directory as the main LaTeX file. For loading figures
%% from other directories you can use the `import` package
%%   \usepackage{import}
%%
%% and then include the figures with
%%   \import{<path to file>}{<filename>.pgf}
%%
%% Matplotlib used the following preamble
%%   
%%   \makeatletter\@ifpackageloaded{underscore}{}{\usepackage[strings]{underscore}}\makeatother
%%
\begingroup%
\makeatletter%
\begin{pgfpicture}%
\pgfpathrectangle{\pgfpointorigin}{\pgfqpoint{3.195241in}{4.613641in}}%
\pgfusepath{use as bounding box, clip}%
\begin{pgfscope}%
\pgfsetbuttcap%
\pgfsetmiterjoin%
\definecolor{currentfill}{rgb}{1.000000,1.000000,1.000000}%
\pgfsetfillcolor{currentfill}%
\pgfsetlinewidth{0.000000pt}%
\definecolor{currentstroke}{rgb}{1.000000,1.000000,1.000000}%
\pgfsetstrokecolor{currentstroke}%
\pgfsetdash{}{0pt}%
\pgfpathmoveto{\pgfqpoint{0.000000in}{0.000000in}}%
\pgfpathlineto{\pgfqpoint{3.195241in}{0.000000in}}%
\pgfpathlineto{\pgfqpoint{3.195241in}{4.613641in}}%
\pgfpathlineto{\pgfqpoint{0.000000in}{4.613641in}}%
\pgfpathlineto{\pgfqpoint{0.000000in}{0.000000in}}%
\pgfpathclose%
\pgfusepath{fill}%
\end{pgfscope}%
\begin{pgfscope}%
\pgfsetbuttcap%
\pgfsetmiterjoin%
\definecolor{currentfill}{rgb}{1.000000,1.000000,1.000000}%
\pgfsetfillcolor{currentfill}%
\pgfsetlinewidth{0.000000pt}%
\definecolor{currentstroke}{rgb}{0.000000,0.000000,0.000000}%
\pgfsetstrokecolor{currentstroke}%
\pgfsetstrokeopacity{0.000000}%
\pgfsetdash{}{0pt}%
\pgfpathmoveto{\pgfqpoint{0.569136in}{0.499691in}}%
\pgfpathlineto{\pgfqpoint{3.095241in}{0.499691in}}%
\pgfpathlineto{\pgfqpoint{3.095241in}{3.548498in}}%
\pgfpathlineto{\pgfqpoint{0.569136in}{3.548498in}}%
\pgfpathlineto{\pgfqpoint{0.569136in}{0.499691in}}%
\pgfpathclose%
\pgfusepath{fill}%
\end{pgfscope}%
\begin{pgfscope}%
\pgfsetbuttcap%
\pgfsetroundjoin%
\definecolor{currentfill}{rgb}{0.000000,0.000000,0.000000}%
\pgfsetfillcolor{currentfill}%
\pgfsetlinewidth{0.803000pt}%
\definecolor{currentstroke}{rgb}{0.000000,0.000000,0.000000}%
\pgfsetstrokecolor{currentstroke}%
\pgfsetdash{}{0pt}%
\pgfsys@defobject{currentmarker}{\pgfqpoint{0.000000in}{-0.048611in}}{\pgfqpoint{0.000000in}{0.000000in}}{%
\pgfpathmoveto{\pgfqpoint{0.000000in}{0.000000in}}%
\pgfpathlineto{\pgfqpoint{0.000000in}{-0.048611in}}%
\pgfusepath{stroke,fill}%
}%
\begin{pgfscope}%
\pgfsys@transformshift{0.683959in}{0.499691in}%
\pgfsys@useobject{currentmarker}{}%
\end{pgfscope}%
\end{pgfscope}%
\begin{pgfscope}%
\definecolor{textcolor}{rgb}{0.000000,0.000000,0.000000}%
\pgfsetstrokecolor{textcolor}%
\pgfsetfillcolor{textcolor}%
\pgftext[x=0.683959in,y=0.402469in,,top]{\color{textcolor}\rmfamily\fontsize{10.000000}{12.000000}\selectfont \(\displaystyle {1}\)}%
\end{pgfscope}%
\begin{pgfscope}%
\pgfsetbuttcap%
\pgfsetroundjoin%
\definecolor{currentfill}{rgb}{0.000000,0.000000,0.000000}%
\pgfsetfillcolor{currentfill}%
\pgfsetlinewidth{0.803000pt}%
\definecolor{currentstroke}{rgb}{0.000000,0.000000,0.000000}%
\pgfsetstrokecolor{currentstroke}%
\pgfsetdash{}{0pt}%
\pgfsys@defobject{currentmarker}{\pgfqpoint{0.000000in}{-0.048611in}}{\pgfqpoint{0.000000in}{0.000000in}}{%
\pgfpathmoveto{\pgfqpoint{0.000000in}{0.000000in}}%
\pgfpathlineto{\pgfqpoint{0.000000in}{-0.048611in}}%
\pgfusepath{stroke,fill}%
}%
\begin{pgfscope}%
\pgfsys@transformshift{0.971017in}{0.499691in}%
\pgfsys@useobject{currentmarker}{}%
\end{pgfscope}%
\end{pgfscope}%
\begin{pgfscope}%
\definecolor{textcolor}{rgb}{0.000000,0.000000,0.000000}%
\pgfsetstrokecolor{textcolor}%
\pgfsetfillcolor{textcolor}%
\pgftext[x=0.971017in,y=0.402469in,,top]{\color{textcolor}\rmfamily\fontsize{10.000000}{12.000000}\selectfont \(\displaystyle {2}\)}%
\end{pgfscope}%
\begin{pgfscope}%
\pgfsetbuttcap%
\pgfsetroundjoin%
\definecolor{currentfill}{rgb}{0.000000,0.000000,0.000000}%
\pgfsetfillcolor{currentfill}%
\pgfsetlinewidth{0.803000pt}%
\definecolor{currentstroke}{rgb}{0.000000,0.000000,0.000000}%
\pgfsetstrokecolor{currentstroke}%
\pgfsetdash{}{0pt}%
\pgfsys@defobject{currentmarker}{\pgfqpoint{0.000000in}{-0.048611in}}{\pgfqpoint{0.000000in}{0.000000in}}{%
\pgfpathmoveto{\pgfqpoint{0.000000in}{0.000000in}}%
\pgfpathlineto{\pgfqpoint{0.000000in}{-0.048611in}}%
\pgfusepath{stroke,fill}%
}%
\begin{pgfscope}%
\pgfsys@transformshift{1.258074in}{0.499691in}%
\pgfsys@useobject{currentmarker}{}%
\end{pgfscope}%
\end{pgfscope}%
\begin{pgfscope}%
\definecolor{textcolor}{rgb}{0.000000,0.000000,0.000000}%
\pgfsetstrokecolor{textcolor}%
\pgfsetfillcolor{textcolor}%
\pgftext[x=1.258074in,y=0.402469in,,top]{\color{textcolor}\rmfamily\fontsize{10.000000}{12.000000}\selectfont \(\displaystyle {3}\)}%
\end{pgfscope}%
\begin{pgfscope}%
\pgfsetbuttcap%
\pgfsetroundjoin%
\definecolor{currentfill}{rgb}{0.000000,0.000000,0.000000}%
\pgfsetfillcolor{currentfill}%
\pgfsetlinewidth{0.803000pt}%
\definecolor{currentstroke}{rgb}{0.000000,0.000000,0.000000}%
\pgfsetstrokecolor{currentstroke}%
\pgfsetdash{}{0pt}%
\pgfsys@defobject{currentmarker}{\pgfqpoint{0.000000in}{-0.048611in}}{\pgfqpoint{0.000000in}{0.000000in}}{%
\pgfpathmoveto{\pgfqpoint{0.000000in}{0.000000in}}%
\pgfpathlineto{\pgfqpoint{0.000000in}{-0.048611in}}%
\pgfusepath{stroke,fill}%
}%
\begin{pgfscope}%
\pgfsys@transformshift{1.545131in}{0.499691in}%
\pgfsys@useobject{currentmarker}{}%
\end{pgfscope}%
\end{pgfscope}%
\begin{pgfscope}%
\definecolor{textcolor}{rgb}{0.000000,0.000000,0.000000}%
\pgfsetstrokecolor{textcolor}%
\pgfsetfillcolor{textcolor}%
\pgftext[x=1.545131in,y=0.402469in,,top]{\color{textcolor}\rmfamily\fontsize{10.000000}{12.000000}\selectfont \(\displaystyle {4}\)}%
\end{pgfscope}%
\begin{pgfscope}%
\pgfsetbuttcap%
\pgfsetroundjoin%
\definecolor{currentfill}{rgb}{0.000000,0.000000,0.000000}%
\pgfsetfillcolor{currentfill}%
\pgfsetlinewidth{0.803000pt}%
\definecolor{currentstroke}{rgb}{0.000000,0.000000,0.000000}%
\pgfsetstrokecolor{currentstroke}%
\pgfsetdash{}{0pt}%
\pgfsys@defobject{currentmarker}{\pgfqpoint{0.000000in}{-0.048611in}}{\pgfqpoint{0.000000in}{0.000000in}}{%
\pgfpathmoveto{\pgfqpoint{0.000000in}{0.000000in}}%
\pgfpathlineto{\pgfqpoint{0.000000in}{-0.048611in}}%
\pgfusepath{stroke,fill}%
}%
\begin{pgfscope}%
\pgfsys@transformshift{1.832189in}{0.499691in}%
\pgfsys@useobject{currentmarker}{}%
\end{pgfscope}%
\end{pgfscope}%
\begin{pgfscope}%
\definecolor{textcolor}{rgb}{0.000000,0.000000,0.000000}%
\pgfsetstrokecolor{textcolor}%
\pgfsetfillcolor{textcolor}%
\pgftext[x=1.832189in,y=0.402469in,,top]{\color{textcolor}\rmfamily\fontsize{10.000000}{12.000000}\selectfont \(\displaystyle {5}\)}%
\end{pgfscope}%
\begin{pgfscope}%
\pgfsetbuttcap%
\pgfsetroundjoin%
\definecolor{currentfill}{rgb}{0.000000,0.000000,0.000000}%
\pgfsetfillcolor{currentfill}%
\pgfsetlinewidth{0.803000pt}%
\definecolor{currentstroke}{rgb}{0.000000,0.000000,0.000000}%
\pgfsetstrokecolor{currentstroke}%
\pgfsetdash{}{0pt}%
\pgfsys@defobject{currentmarker}{\pgfqpoint{0.000000in}{-0.048611in}}{\pgfqpoint{0.000000in}{0.000000in}}{%
\pgfpathmoveto{\pgfqpoint{0.000000in}{0.000000in}}%
\pgfpathlineto{\pgfqpoint{0.000000in}{-0.048611in}}%
\pgfusepath{stroke,fill}%
}%
\begin{pgfscope}%
\pgfsys@transformshift{2.119246in}{0.499691in}%
\pgfsys@useobject{currentmarker}{}%
\end{pgfscope}%
\end{pgfscope}%
\begin{pgfscope}%
\definecolor{textcolor}{rgb}{0.000000,0.000000,0.000000}%
\pgfsetstrokecolor{textcolor}%
\pgfsetfillcolor{textcolor}%
\pgftext[x=2.119246in,y=0.402469in,,top]{\color{textcolor}\rmfamily\fontsize{10.000000}{12.000000}\selectfont \(\displaystyle {6}\)}%
\end{pgfscope}%
\begin{pgfscope}%
\pgfsetbuttcap%
\pgfsetroundjoin%
\definecolor{currentfill}{rgb}{0.000000,0.000000,0.000000}%
\pgfsetfillcolor{currentfill}%
\pgfsetlinewidth{0.803000pt}%
\definecolor{currentstroke}{rgb}{0.000000,0.000000,0.000000}%
\pgfsetstrokecolor{currentstroke}%
\pgfsetdash{}{0pt}%
\pgfsys@defobject{currentmarker}{\pgfqpoint{0.000000in}{-0.048611in}}{\pgfqpoint{0.000000in}{0.000000in}}{%
\pgfpathmoveto{\pgfqpoint{0.000000in}{0.000000in}}%
\pgfpathlineto{\pgfqpoint{0.000000in}{-0.048611in}}%
\pgfusepath{stroke,fill}%
}%
\begin{pgfscope}%
\pgfsys@transformshift{2.406303in}{0.499691in}%
\pgfsys@useobject{currentmarker}{}%
\end{pgfscope}%
\end{pgfscope}%
\begin{pgfscope}%
\definecolor{textcolor}{rgb}{0.000000,0.000000,0.000000}%
\pgfsetstrokecolor{textcolor}%
\pgfsetfillcolor{textcolor}%
\pgftext[x=2.406303in,y=0.402469in,,top]{\color{textcolor}\rmfamily\fontsize{10.000000}{12.000000}\selectfont \(\displaystyle {7}\)}%
\end{pgfscope}%
\begin{pgfscope}%
\pgfsetbuttcap%
\pgfsetroundjoin%
\definecolor{currentfill}{rgb}{0.000000,0.000000,0.000000}%
\pgfsetfillcolor{currentfill}%
\pgfsetlinewidth{0.803000pt}%
\definecolor{currentstroke}{rgb}{0.000000,0.000000,0.000000}%
\pgfsetstrokecolor{currentstroke}%
\pgfsetdash{}{0pt}%
\pgfsys@defobject{currentmarker}{\pgfqpoint{0.000000in}{-0.048611in}}{\pgfqpoint{0.000000in}{0.000000in}}{%
\pgfpathmoveto{\pgfqpoint{0.000000in}{0.000000in}}%
\pgfpathlineto{\pgfqpoint{0.000000in}{-0.048611in}}%
\pgfusepath{stroke,fill}%
}%
\begin{pgfscope}%
\pgfsys@transformshift{2.693361in}{0.499691in}%
\pgfsys@useobject{currentmarker}{}%
\end{pgfscope}%
\end{pgfscope}%
\begin{pgfscope}%
\definecolor{textcolor}{rgb}{0.000000,0.000000,0.000000}%
\pgfsetstrokecolor{textcolor}%
\pgfsetfillcolor{textcolor}%
\pgftext[x=2.693361in,y=0.402469in,,top]{\color{textcolor}\rmfamily\fontsize{10.000000}{12.000000}\selectfont \(\displaystyle {8}\)}%
\end{pgfscope}%
\begin{pgfscope}%
\pgfsetbuttcap%
\pgfsetroundjoin%
\definecolor{currentfill}{rgb}{0.000000,0.000000,0.000000}%
\pgfsetfillcolor{currentfill}%
\pgfsetlinewidth{0.803000pt}%
\definecolor{currentstroke}{rgb}{0.000000,0.000000,0.000000}%
\pgfsetstrokecolor{currentstroke}%
\pgfsetdash{}{0pt}%
\pgfsys@defobject{currentmarker}{\pgfqpoint{0.000000in}{-0.048611in}}{\pgfqpoint{0.000000in}{0.000000in}}{%
\pgfpathmoveto{\pgfqpoint{0.000000in}{0.000000in}}%
\pgfpathlineto{\pgfqpoint{0.000000in}{-0.048611in}}%
\pgfusepath{stroke,fill}%
}%
\begin{pgfscope}%
\pgfsys@transformshift{2.980418in}{0.499691in}%
\pgfsys@useobject{currentmarker}{}%
\end{pgfscope}%
\end{pgfscope}%
\begin{pgfscope}%
\definecolor{textcolor}{rgb}{0.000000,0.000000,0.000000}%
\pgfsetstrokecolor{textcolor}%
\pgfsetfillcolor{textcolor}%
\pgftext[x=2.980418in,y=0.402469in,,top]{\color{textcolor}\rmfamily\fontsize{10.000000}{12.000000}\selectfont \(\displaystyle {9}\)}%
\end{pgfscope}%
\begin{pgfscope}%
\definecolor{textcolor}{rgb}{0.000000,0.000000,0.000000}%
\pgfsetstrokecolor{textcolor}%
\pgfsetfillcolor{textcolor}%
\pgftext[x=1.832189in,y=0.223457in,,top]{\color{textcolor}\rmfamily\fontsize{10.000000}{12.000000}\selectfont gzip compression level}%
\end{pgfscope}%
\begin{pgfscope}%
\pgfsetbuttcap%
\pgfsetroundjoin%
\definecolor{currentfill}{rgb}{0.000000,0.000000,0.000000}%
\pgfsetfillcolor{currentfill}%
\pgfsetlinewidth{0.803000pt}%
\definecolor{currentstroke}{rgb}{0.000000,0.000000,0.000000}%
\pgfsetstrokecolor{currentstroke}%
\pgfsetdash{}{0pt}%
\pgfsys@defobject{currentmarker}{\pgfqpoint{-0.048611in}{0.000000in}}{\pgfqpoint{-0.000000in}{0.000000in}}{%
\pgfpathmoveto{\pgfqpoint{-0.000000in}{0.000000in}}%
\pgfpathlineto{\pgfqpoint{-0.048611in}{0.000000in}}%
\pgfusepath{stroke,fill}%
}%
\begin{pgfscope}%
\pgfsys@transformshift{0.569136in}{0.713183in}%
\pgfsys@useobject{currentmarker}{}%
\end{pgfscope}%
\end{pgfscope}%
\begin{pgfscope}%
\definecolor{textcolor}{rgb}{0.000000,0.000000,0.000000}%
\pgfsetstrokecolor{textcolor}%
\pgfsetfillcolor{textcolor}%
\pgftext[x=0.294444in, y=0.664957in, left, base]{\color{textcolor}\rmfamily\fontsize{10.000000}{12.000000}\selectfont \(\displaystyle {0.3}\)}%
\end{pgfscope}%
\begin{pgfscope}%
\pgfsetbuttcap%
\pgfsetroundjoin%
\definecolor{currentfill}{rgb}{0.000000,0.000000,0.000000}%
\pgfsetfillcolor{currentfill}%
\pgfsetlinewidth{0.803000pt}%
\definecolor{currentstroke}{rgb}{0.000000,0.000000,0.000000}%
\pgfsetstrokecolor{currentstroke}%
\pgfsetdash{}{0pt}%
\pgfsys@defobject{currentmarker}{\pgfqpoint{-0.048611in}{0.000000in}}{\pgfqpoint{-0.000000in}{0.000000in}}{%
\pgfpathmoveto{\pgfqpoint{-0.000000in}{0.000000in}}%
\pgfpathlineto{\pgfqpoint{-0.048611in}{0.000000in}}%
\pgfusepath{stroke,fill}%
}%
\begin{pgfscope}%
\pgfsys@transformshift{0.569136in}{1.181365in}%
\pgfsys@useobject{currentmarker}{}%
\end{pgfscope}%
\end{pgfscope}%
\begin{pgfscope}%
\definecolor{textcolor}{rgb}{0.000000,0.000000,0.000000}%
\pgfsetstrokecolor{textcolor}%
\pgfsetfillcolor{textcolor}%
\pgftext[x=0.294444in, y=1.133140in, left, base]{\color{textcolor}\rmfamily\fontsize{10.000000}{12.000000}\selectfont \(\displaystyle {0.4}\)}%
\end{pgfscope}%
\begin{pgfscope}%
\pgfsetbuttcap%
\pgfsetroundjoin%
\definecolor{currentfill}{rgb}{0.000000,0.000000,0.000000}%
\pgfsetfillcolor{currentfill}%
\pgfsetlinewidth{0.803000pt}%
\definecolor{currentstroke}{rgb}{0.000000,0.000000,0.000000}%
\pgfsetstrokecolor{currentstroke}%
\pgfsetdash{}{0pt}%
\pgfsys@defobject{currentmarker}{\pgfqpoint{-0.048611in}{0.000000in}}{\pgfqpoint{-0.000000in}{0.000000in}}{%
\pgfpathmoveto{\pgfqpoint{-0.000000in}{0.000000in}}%
\pgfpathlineto{\pgfqpoint{-0.048611in}{0.000000in}}%
\pgfusepath{stroke,fill}%
}%
\begin{pgfscope}%
\pgfsys@transformshift{0.569136in}{1.649548in}%
\pgfsys@useobject{currentmarker}{}%
\end{pgfscope}%
\end{pgfscope}%
\begin{pgfscope}%
\definecolor{textcolor}{rgb}{0.000000,0.000000,0.000000}%
\pgfsetstrokecolor{textcolor}%
\pgfsetfillcolor{textcolor}%
\pgftext[x=0.294444in, y=1.601323in, left, base]{\color{textcolor}\rmfamily\fontsize{10.000000}{12.000000}\selectfont \(\displaystyle {0.5}\)}%
\end{pgfscope}%
\begin{pgfscope}%
\pgfsetbuttcap%
\pgfsetroundjoin%
\definecolor{currentfill}{rgb}{0.000000,0.000000,0.000000}%
\pgfsetfillcolor{currentfill}%
\pgfsetlinewidth{0.803000pt}%
\definecolor{currentstroke}{rgb}{0.000000,0.000000,0.000000}%
\pgfsetstrokecolor{currentstroke}%
\pgfsetdash{}{0pt}%
\pgfsys@defobject{currentmarker}{\pgfqpoint{-0.048611in}{0.000000in}}{\pgfqpoint{-0.000000in}{0.000000in}}{%
\pgfpathmoveto{\pgfqpoint{-0.000000in}{0.000000in}}%
\pgfpathlineto{\pgfqpoint{-0.048611in}{0.000000in}}%
\pgfusepath{stroke,fill}%
}%
\begin{pgfscope}%
\pgfsys@transformshift{0.569136in}{2.117731in}%
\pgfsys@useobject{currentmarker}{}%
\end{pgfscope}%
\end{pgfscope}%
\begin{pgfscope}%
\definecolor{textcolor}{rgb}{0.000000,0.000000,0.000000}%
\pgfsetstrokecolor{textcolor}%
\pgfsetfillcolor{textcolor}%
\pgftext[x=0.294444in, y=2.069506in, left, base]{\color{textcolor}\rmfamily\fontsize{10.000000}{12.000000}\selectfont \(\displaystyle {0.6}\)}%
\end{pgfscope}%
\begin{pgfscope}%
\pgfsetbuttcap%
\pgfsetroundjoin%
\definecolor{currentfill}{rgb}{0.000000,0.000000,0.000000}%
\pgfsetfillcolor{currentfill}%
\pgfsetlinewidth{0.803000pt}%
\definecolor{currentstroke}{rgb}{0.000000,0.000000,0.000000}%
\pgfsetstrokecolor{currentstroke}%
\pgfsetdash{}{0pt}%
\pgfsys@defobject{currentmarker}{\pgfqpoint{-0.048611in}{0.000000in}}{\pgfqpoint{-0.000000in}{0.000000in}}{%
\pgfpathmoveto{\pgfqpoint{-0.000000in}{0.000000in}}%
\pgfpathlineto{\pgfqpoint{-0.048611in}{0.000000in}}%
\pgfusepath{stroke,fill}%
}%
\begin{pgfscope}%
\pgfsys@transformshift{0.569136in}{2.585914in}%
\pgfsys@useobject{currentmarker}{}%
\end{pgfscope}%
\end{pgfscope}%
\begin{pgfscope}%
\definecolor{textcolor}{rgb}{0.000000,0.000000,0.000000}%
\pgfsetstrokecolor{textcolor}%
\pgfsetfillcolor{textcolor}%
\pgftext[x=0.294444in, y=2.537689in, left, base]{\color{textcolor}\rmfamily\fontsize{10.000000}{12.000000}\selectfont \(\displaystyle {0.7}\)}%
\end{pgfscope}%
\begin{pgfscope}%
\pgfsetbuttcap%
\pgfsetroundjoin%
\definecolor{currentfill}{rgb}{0.000000,0.000000,0.000000}%
\pgfsetfillcolor{currentfill}%
\pgfsetlinewidth{0.803000pt}%
\definecolor{currentstroke}{rgb}{0.000000,0.000000,0.000000}%
\pgfsetstrokecolor{currentstroke}%
\pgfsetdash{}{0pt}%
\pgfsys@defobject{currentmarker}{\pgfqpoint{-0.048611in}{0.000000in}}{\pgfqpoint{-0.000000in}{0.000000in}}{%
\pgfpathmoveto{\pgfqpoint{-0.000000in}{0.000000in}}%
\pgfpathlineto{\pgfqpoint{-0.048611in}{0.000000in}}%
\pgfusepath{stroke,fill}%
}%
\begin{pgfscope}%
\pgfsys@transformshift{0.569136in}{3.054097in}%
\pgfsys@useobject{currentmarker}{}%
\end{pgfscope}%
\end{pgfscope}%
\begin{pgfscope}%
\definecolor{textcolor}{rgb}{0.000000,0.000000,0.000000}%
\pgfsetstrokecolor{textcolor}%
\pgfsetfillcolor{textcolor}%
\pgftext[x=0.294444in, y=3.005872in, left, base]{\color{textcolor}\rmfamily\fontsize{10.000000}{12.000000}\selectfont \(\displaystyle {0.8}\)}%
\end{pgfscope}%
\begin{pgfscope}%
\pgfsetbuttcap%
\pgfsetroundjoin%
\definecolor{currentfill}{rgb}{0.000000,0.000000,0.000000}%
\pgfsetfillcolor{currentfill}%
\pgfsetlinewidth{0.803000pt}%
\definecolor{currentstroke}{rgb}{0.000000,0.000000,0.000000}%
\pgfsetstrokecolor{currentstroke}%
\pgfsetdash{}{0pt}%
\pgfsys@defobject{currentmarker}{\pgfqpoint{-0.048611in}{0.000000in}}{\pgfqpoint{-0.000000in}{0.000000in}}{%
\pgfpathmoveto{\pgfqpoint{-0.000000in}{0.000000in}}%
\pgfpathlineto{\pgfqpoint{-0.048611in}{0.000000in}}%
\pgfusepath{stroke,fill}%
}%
\begin{pgfscope}%
\pgfsys@transformshift{0.569136in}{3.522280in}%
\pgfsys@useobject{currentmarker}{}%
\end{pgfscope}%
\end{pgfscope}%
\begin{pgfscope}%
\definecolor{textcolor}{rgb}{0.000000,0.000000,0.000000}%
\pgfsetstrokecolor{textcolor}%
\pgfsetfillcolor{textcolor}%
\pgftext[x=0.294444in, y=3.474055in, left, base]{\color{textcolor}\rmfamily\fontsize{10.000000}{12.000000}\selectfont \(\displaystyle {0.9}\)}%
\end{pgfscope}%
\begin{pgfscope}%
\definecolor{textcolor}{rgb}{0.000000,0.000000,0.000000}%
\pgfsetstrokecolor{textcolor}%
\pgfsetfillcolor{textcolor}%
\pgftext[x=0.238889in,y=2.024095in,,bottom,rotate=90.000000]{\color{textcolor}\rmfamily\fontsize{10.000000}{12.000000}\selectfont decompression time (s)}%
\end{pgfscope}%
\begin{pgfscope}%
\pgfpathrectangle{\pgfqpoint{0.569136in}{0.499691in}}{\pgfqpoint{2.526105in}{3.048807in}}%
\pgfusepath{clip}%
\pgfsetrectcap%
\pgfsetroundjoin%
\pgfsetlinewidth{1.505625pt}%
\definecolor{currentstroke}{rgb}{0.839216,0.152941,0.156863}%
\pgfsetstrokecolor{currentstroke}%
\pgfsetdash{}{0pt}%
\pgfpathmoveto{\pgfqpoint{0.683959in}{2.951097in}}%
\pgfpathlineto{\pgfqpoint{0.971017in}{3.007279in}}%
\pgfpathlineto{\pgfqpoint{1.258074in}{2.969824in}}%
\pgfpathlineto{\pgfqpoint{1.545131in}{3.306916in}}%
\pgfpathlineto{\pgfqpoint{1.832189in}{3.316280in}}%
\pgfpathlineto{\pgfqpoint{2.119246in}{3.316280in}}%
\pgfpathlineto{\pgfqpoint{2.406303in}{3.353734in}}%
\pgfpathlineto{\pgfqpoint{2.693361in}{3.409916in}}%
\pgfpathlineto{\pgfqpoint{2.980418in}{3.363098in}}%
\pgfusepath{stroke}%
\end{pgfscope}%
\begin{pgfscope}%
\pgfpathrectangle{\pgfqpoint{0.569136in}{0.499691in}}{\pgfqpoint{2.526105in}{3.048807in}}%
\pgfusepath{clip}%
\pgfsetrectcap%
\pgfsetroundjoin%
\pgfsetlinewidth{1.505625pt}%
\definecolor{currentstroke}{rgb}{0.172549,0.627451,0.172549}%
\pgfsetstrokecolor{currentstroke}%
\pgfsetdash{}{0pt}%
\pgfpathmoveto{\pgfqpoint{0.683959in}{2.969824in}}%
\pgfpathlineto{\pgfqpoint{0.971017in}{2.997915in}}%
\pgfpathlineto{\pgfqpoint{1.258074in}{3.016643in}}%
\pgfpathlineto{\pgfqpoint{1.545131in}{2.576551in}}%
\pgfpathlineto{\pgfqpoint{1.832189in}{2.548460in}}%
\pgfpathlineto{\pgfqpoint{2.119246in}{2.548460in}}%
\pgfpathlineto{\pgfqpoint{2.406303in}{2.548460in}}%
\pgfpathlineto{\pgfqpoint{2.693361in}{2.567187in}}%
\pgfpathlineto{\pgfqpoint{2.980418in}{2.548460in}}%
\pgfusepath{stroke}%
\end{pgfscope}%
\begin{pgfscope}%
\pgfpathrectangle{\pgfqpoint{0.569136in}{0.499691in}}{\pgfqpoint{2.526105in}{3.048807in}}%
\pgfusepath{clip}%
\pgfsetrectcap%
\pgfsetroundjoin%
\pgfsetlinewidth{1.505625pt}%
\definecolor{currentstroke}{rgb}{0.580392,0.403922,0.741176}%
\pgfsetstrokecolor{currentstroke}%
\pgfsetdash{}{0pt}%
\pgfpathmoveto{\pgfqpoint{0.683959in}{2.960461in}}%
\pgfpathlineto{\pgfqpoint{0.971017in}{2.969824in}}%
\pgfpathlineto{\pgfqpoint{1.258074in}{3.026006in}}%
\pgfpathlineto{\pgfqpoint{1.545131in}{2.548460in}}%
\pgfpathlineto{\pgfqpoint{1.832189in}{2.511005in}}%
\pgfpathlineto{\pgfqpoint{2.119246in}{2.482914in}}%
\pgfpathlineto{\pgfqpoint{2.406303in}{2.548460in}}%
\pgfpathlineto{\pgfqpoint{2.693361in}{2.482914in}}%
\pgfpathlineto{\pgfqpoint{2.980418in}{2.464187in}}%
\pgfusepath{stroke}%
\end{pgfscope}%
\begin{pgfscope}%
\pgfpathrectangle{\pgfqpoint{0.569136in}{0.499691in}}{\pgfqpoint{2.526105in}{3.048807in}}%
\pgfusepath{clip}%
\pgfsetrectcap%
\pgfsetroundjoin%
\pgfsetlinewidth{1.505625pt}%
\definecolor{currentstroke}{rgb}{1.000000,0.498039,0.054902}%
\pgfsetstrokecolor{currentstroke}%
\pgfsetdash{}{0pt}%
\pgfpathmoveto{\pgfqpoint{0.683959in}{1.378002in}}%
\pgfpathlineto{\pgfqpoint{0.971017in}{1.321820in}}%
\pgfpathlineto{\pgfqpoint{1.258074in}{1.331184in}}%
\pgfpathlineto{\pgfqpoint{1.545131in}{1.303093in}}%
\pgfpathlineto{\pgfqpoint{1.832189in}{1.312457in}}%
\pgfpathlineto{\pgfqpoint{2.119246in}{1.331184in}}%
\pgfpathlineto{\pgfqpoint{2.406303in}{1.293729in}}%
\pgfpathlineto{\pgfqpoint{2.693361in}{1.312457in}}%
\pgfpathlineto{\pgfqpoint{2.980418in}{1.312457in}}%
\pgfusepath{stroke}%
\end{pgfscope}%
\begin{pgfscope}%
\pgfpathrectangle{\pgfqpoint{0.569136in}{0.499691in}}{\pgfqpoint{2.526105in}{3.048807in}}%
\pgfusepath{clip}%
\pgfsetrectcap%
\pgfsetroundjoin%
\pgfsetlinewidth{1.505625pt}%
\definecolor{currentstroke}{rgb}{0.549020,0.337255,0.294118}%
\pgfsetstrokecolor{currentstroke}%
\pgfsetdash{}{0pt}%
\pgfpathmoveto{\pgfqpoint{0.683959in}{1.331184in}}%
\pgfpathlineto{\pgfqpoint{0.971017in}{1.340548in}}%
\pgfpathlineto{\pgfqpoint{1.258074in}{1.359275in}}%
\pgfpathlineto{\pgfqpoint{1.545131in}{1.349911in}}%
\pgfpathlineto{\pgfqpoint{1.832189in}{1.303093in}}%
\pgfpathlineto{\pgfqpoint{2.119246in}{1.303093in}}%
\pgfpathlineto{\pgfqpoint{2.406303in}{1.340548in}}%
\pgfpathlineto{\pgfqpoint{2.693361in}{1.321820in}}%
\pgfpathlineto{\pgfqpoint{2.980418in}{1.321820in}}%
\pgfusepath{stroke}%
\end{pgfscope}%
\begin{pgfscope}%
\pgfpathrectangle{\pgfqpoint{0.569136in}{0.499691in}}{\pgfqpoint{2.526105in}{3.048807in}}%
\pgfusepath{clip}%
\pgfsetrectcap%
\pgfsetroundjoin%
\pgfsetlinewidth{1.505625pt}%
\definecolor{currentstroke}{rgb}{0.121569,0.466667,0.705882}%
\pgfsetstrokecolor{currentstroke}%
\pgfsetdash{}{0pt}%
\pgfpathmoveto{\pgfqpoint{0.683959in}{0.685092in}}%
\pgfpathlineto{\pgfqpoint{0.971017in}{0.713183in}}%
\pgfpathlineto{\pgfqpoint{1.258074in}{0.675728in}}%
\pgfpathlineto{\pgfqpoint{1.545131in}{0.647637in}}%
\pgfpathlineto{\pgfqpoint{1.832189in}{0.657001in}}%
\pgfpathlineto{\pgfqpoint{2.119246in}{0.666364in}}%
\pgfpathlineto{\pgfqpoint{2.406303in}{0.675728in}}%
\pgfpathlineto{\pgfqpoint{2.693361in}{0.638273in}}%
\pgfpathlineto{\pgfqpoint{2.980418in}{0.694455in}}%
\pgfusepath{stroke}%
\end{pgfscope}%
\begin{pgfscope}%
\pgfpathrectangle{\pgfqpoint{0.569136in}{0.499691in}}{\pgfqpoint{2.526105in}{3.048807in}}%
\pgfusepath{clip}%
\pgfsetbuttcap%
\pgfsetroundjoin%
\pgfsetlinewidth{1.505625pt}%
\definecolor{currentstroke}{rgb}{0.000000,0.000000,0.000000}%
\pgfsetstrokecolor{currentstroke}%
\pgfsetstrokeopacity{0.500000}%
\pgfsetdash{{5.550000pt}{2.400000pt}}{0.000000pt}%
\pgfpathmoveto{\pgfqpoint{2.119246in}{0.499691in}}%
\pgfpathlineto{\pgfqpoint{2.119246in}{3.548498in}}%
\pgfusepath{stroke}%
\end{pgfscope}%
\begin{pgfscope}%
\pgfsetrectcap%
\pgfsetmiterjoin%
\pgfsetlinewidth{0.803000pt}%
\definecolor{currentstroke}{rgb}{0.000000,0.000000,0.000000}%
\pgfsetstrokecolor{currentstroke}%
\pgfsetdash{}{0pt}%
\pgfpathmoveto{\pgfqpoint{0.569136in}{0.499691in}}%
\pgfpathlineto{\pgfqpoint{0.569136in}{3.548498in}}%
\pgfusepath{stroke}%
\end{pgfscope}%
\begin{pgfscope}%
\pgfsetrectcap%
\pgfsetmiterjoin%
\pgfsetlinewidth{0.803000pt}%
\definecolor{currentstroke}{rgb}{0.000000,0.000000,0.000000}%
\pgfsetstrokecolor{currentstroke}%
\pgfsetdash{}{0pt}%
\pgfpathmoveto{\pgfqpoint{3.095241in}{0.499691in}}%
\pgfpathlineto{\pgfqpoint{3.095241in}{3.548498in}}%
\pgfusepath{stroke}%
\end{pgfscope}%
\begin{pgfscope}%
\pgfsetrectcap%
\pgfsetmiterjoin%
\pgfsetlinewidth{0.803000pt}%
\definecolor{currentstroke}{rgb}{0.000000,0.000000,0.000000}%
\pgfsetstrokecolor{currentstroke}%
\pgfsetdash{}{0pt}%
\pgfpathmoveto{\pgfqpoint{0.569136in}{0.499691in}}%
\pgfpathlineto{\pgfqpoint{3.095241in}{0.499691in}}%
\pgfusepath{stroke}%
\end{pgfscope}%
\begin{pgfscope}%
\pgfsetrectcap%
\pgfsetmiterjoin%
\pgfsetlinewidth{0.803000pt}%
\definecolor{currentstroke}{rgb}{0.000000,0.000000,0.000000}%
\pgfsetstrokecolor{currentstroke}%
\pgfsetdash{}{0pt}%
\pgfpathmoveto{\pgfqpoint{0.569136in}{3.548498in}}%
\pgfpathlineto{\pgfqpoint{3.095241in}{3.548498in}}%
\pgfusepath{stroke}%
\end{pgfscope}%
\begin{pgfscope}%
\pgfsetbuttcap%
\pgfsetmiterjoin%
\definecolor{currentfill}{rgb}{1.000000,1.000000,1.000000}%
\pgfsetfillcolor{currentfill}%
\pgfsetfillopacity{0.800000}%
\pgfsetlinewidth{1.003750pt}%
\definecolor{currentstroke}{rgb}{0.800000,0.800000,0.800000}%
\pgfsetstrokecolor{currentstroke}%
\pgfsetstrokeopacity{0.800000}%
\pgfsetdash{}{0pt}%
\pgfpathmoveto{\pgfqpoint{0.588581in}{3.609475in}}%
\pgfpathlineto{\pgfqpoint{3.075797in}{3.609475in}}%
\pgfpathquadraticcurveto{\pgfqpoint{3.095241in}{3.609475in}}{\pgfqpoint{3.095241in}{3.628919in}}%
\pgfpathlineto{\pgfqpoint{3.095241in}{4.494197in}}%
\pgfpathquadraticcurveto{\pgfqpoint{3.095241in}{4.513641in}}{\pgfqpoint{3.075797in}{4.513641in}}%
\pgfpathlineto{\pgfqpoint{0.588581in}{4.513641in}}%
\pgfpathquadraticcurveto{\pgfqpoint{0.569136in}{4.513641in}}{\pgfqpoint{0.569136in}{4.494197in}}%
\pgfpathlineto{\pgfqpoint{0.569136in}{3.628919in}}%
\pgfpathquadraticcurveto{\pgfqpoint{0.569136in}{3.609475in}}{\pgfqpoint{0.588581in}{3.609475in}}%
\pgfpathlineto{\pgfqpoint{0.588581in}{3.609475in}}%
\pgfpathclose%
\pgfusepath{stroke,fill}%
\end{pgfscope}%
\begin{pgfscope}%
\pgfsetrectcap%
\pgfsetroundjoin%
\pgfsetlinewidth{1.505625pt}%
\definecolor{currentstroke}{rgb}{0.839216,0.152941,0.156863}%
\pgfsetstrokecolor{currentstroke}%
\pgfsetdash{}{0pt}%
\pgfpathmoveto{\pgfqpoint{0.608025in}{4.435863in}}%
\pgfpathlineto{\pgfqpoint{0.705248in}{4.435863in}}%
\pgfpathlineto{\pgfqpoint{0.802470in}{4.435863in}}%
\pgfusepath{stroke}%
\end{pgfscope}%
\begin{pgfscope}%
\definecolor{textcolor}{rgb}{0.000000,0.000000,0.000000}%
\pgfsetstrokecolor{textcolor}%
\pgfsetfillcolor{textcolor}%
\pgftext[x=0.880248in,y=4.401836in,left,base]{\color{textcolor}\rmfamily\fontsize{7.000000}{8.400000}\selectfont od --format=x /dev/urandom}%
\end{pgfscope}%
\begin{pgfscope}%
\pgfsetrectcap%
\pgfsetroundjoin%
\pgfsetlinewidth{1.505625pt}%
\definecolor{currentstroke}{rgb}{0.172549,0.627451,0.172549}%
\pgfsetstrokecolor{currentstroke}%
\pgfsetdash{}{0pt}%
\pgfpathmoveto{\pgfqpoint{0.608025in}{4.290030in}}%
\pgfpathlineto{\pgfqpoint{0.705248in}{4.290030in}}%
\pgfpathlineto{\pgfqpoint{0.802470in}{4.290030in}}%
\pgfusepath{stroke}%
\end{pgfscope}%
\begin{pgfscope}%
\definecolor{textcolor}{rgb}{0.000000,0.000000,0.000000}%
\pgfsetstrokecolor{textcolor}%
\pgfsetfillcolor{textcolor}%
\pgftext[x=0.880248in,y=4.256002in,left,base]{\color{textcolor}\rmfamily\fontsize{7.000000}{8.400000}\selectfont base64 /dev/urandom | head -c 1G}%
\end{pgfscope}%
\begin{pgfscope}%
\pgfsetrectcap%
\pgfsetroundjoin%
\pgfsetlinewidth{1.505625pt}%
\definecolor{currentstroke}{rgb}{0.580392,0.403922,0.741176}%
\pgfsetstrokecolor{currentstroke}%
\pgfsetdash{}{0pt}%
\pgfpathmoveto{\pgfqpoint{0.608025in}{4.144197in}}%
\pgfpathlineto{\pgfqpoint{0.705248in}{4.144197in}}%
\pgfpathlineto{\pgfqpoint{0.802470in}{4.144197in}}%
\pgfusepath{stroke}%
\end{pgfscope}%
\begin{pgfscope}%
\definecolor{textcolor}{rgb}{0.000000,0.000000,0.000000}%
\pgfsetstrokecolor{textcolor}%
\pgfsetfillcolor{textcolor}%
\pgftext[x=0.880248in,y=4.110169in,left,base]{\color{textcolor}\rmfamily\fontsize{7.000000}{8.400000}\selectfont cat /dev/urandom | tr -dc 'a-zA-Z0-9' | ...}%
\end{pgfscope}%
\begin{pgfscope}%
\pgfsetrectcap%
\pgfsetroundjoin%
\pgfsetlinewidth{1.505625pt}%
\definecolor{currentstroke}{rgb}{1.000000,0.498039,0.054902}%
\pgfsetstrokecolor{currentstroke}%
\pgfsetdash{}{0pt}%
\pgfpathmoveto{\pgfqpoint{0.608025in}{3.998363in}}%
\pgfpathlineto{\pgfqpoint{0.705248in}{3.998363in}}%
\pgfpathlineto{\pgfqpoint{0.802470in}{3.998363in}}%
\pgfusepath{stroke}%
\end{pgfscope}%
\begin{pgfscope}%
\definecolor{textcolor}{rgb}{0.000000,0.000000,0.000000}%
\pgfsetstrokecolor{textcolor}%
\pgfsetfillcolor{textcolor}%
\pgftext[x=0.880248in,y=3.964336in,left,base]{\color{textcolor}\rmfamily\fontsize{7.000000}{8.400000}\selectfont head -c 1G /dev/urandom}%
\end{pgfscope}%
\begin{pgfscope}%
\pgfsetrectcap%
\pgfsetroundjoin%
\pgfsetlinewidth{1.505625pt}%
\definecolor{currentstroke}{rgb}{0.549020,0.337255,0.294118}%
\pgfsetstrokecolor{currentstroke}%
\pgfsetdash{}{0pt}%
\pgfpathmoveto{\pgfqpoint{0.608025in}{3.852530in}}%
\pgfpathlineto{\pgfqpoint{0.705248in}{3.852530in}}%
\pgfpathlineto{\pgfqpoint{0.802470in}{3.852530in}}%
\pgfusepath{stroke}%
\end{pgfscope}%
\begin{pgfscope}%
\definecolor{textcolor}{rgb}{0.000000,0.000000,0.000000}%
\pgfsetstrokecolor{textcolor}%
\pgfsetfillcolor{textcolor}%
\pgftext[x=0.880248in,y=3.818502in,left,base]{\color{textcolor}\rmfamily\fontsize{7.000000}{8.400000}\selectfont openssl rand -out myfile \$(echo 1G | nu...}%
\end{pgfscope}%
\begin{pgfscope}%
\pgfsetrectcap%
\pgfsetroundjoin%
\pgfsetlinewidth{1.505625pt}%
\definecolor{currentstroke}{rgb}{0.121569,0.466667,0.705882}%
\pgfsetstrokecolor{currentstroke}%
\pgfsetdash{}{0pt}%
\pgfpathmoveto{\pgfqpoint{0.608025in}{3.706697in}}%
\pgfpathlineto{\pgfqpoint{0.705248in}{3.706697in}}%
\pgfpathlineto{\pgfqpoint{0.802470in}{3.706697in}}%
\pgfusepath{stroke}%
\end{pgfscope}%
\begin{pgfscope}%
\definecolor{textcolor}{rgb}{0.000000,0.000000,0.000000}%
\pgfsetstrokecolor{textcolor}%
\pgfsetfillcolor{textcolor}%
\pgftext[x=0.880248in,y=3.672669in,left,base]{\color{textcolor}\rmfamily\fontsize{7.000000}{8.400000}\selectfont ./lorem/lorem -c 1000000000}%
\end{pgfscope}%
\end{pgfpicture}%
\makeatother%
\endgroup%

  \end{center}
  \caption{Decompression times for files generated with popular tools.}
  \label{fig:popular-tools}
\end{figure}

We can see that the decompression times vary significantly
depending on the tool used to generate the file. For example, the file
generated by \texttt{lorem}, containing English text, is decompressed in 3 seconds.
\texttt{openssl} and direct \texttt{/dev/urandom} output are decompressed in 4.5 seconds.
Finally, the files generated by \texttt{base64} and \texttt{tr} require between 7 and 8
seconds to be decompressed, depending on the compression level used. Finally,
\texttt{od}'s output is decompressed in 9 seconds. 

\subsection{\texttt{od}'s output}
\label{sec:org27060b5}

Wishing to understand why \texttt{od}'s output was decompressed slower than the
others, we decided to fully leverage \texttt{od}'s various output formats. We thus
generated files with the following output formats:

\begin{itemize}
\item \texttt{x} (hexadecimal)
\item \texttt{a}, \texttt{c} (ASCII both named and unnamed)
\item \texttt{d1}, \texttt{d2}, \texttt{d4}, \texttt{d8} (decimal)
\item \texttt{f} (floating point)
\item \texttt{o} (octal)
\item \texttt{u1}, \texttt{u2}, \texttt{u4}, \texttt{u8} (unsigned decimal)
\end{itemize}

We decided to use the decimal and unsigned decimal format variants with 1, 2,
4, and 8 bytes in order to see if the size of the numbers (and the minus sign
in decimal formats) had any impact on the decompression time. It must be
remembered we voluntarily un-padded the results of \texttt{od}, and thus, the files
are comprised of a single line of content with no spaces or newlines.

We repeated the same tests as before, and the results are shown in
\autoref{fig:od-output}, although with 100MB files instead of 1GB files, due
to the long time required to generate the files.

\begin{figure}[h!]
  \begin{center}
    %% Creator: Matplotlib, PGF backend
%%
%% To include the figure in your LaTeX document, write
%%   \input{<filename>.pgf}
%%
%% Make sure the required packages are loaded in your preamble
%%   \usepackage{pgf}
%%
%% Also ensure that all the required font packages are loaded; for instance,
%% the lmodern package is sometimes necessary when using math font.
%%   \usepackage{lmodern}
%%
%% Figures using additional raster images can only be included by \input if
%% they are in the same directory as the main LaTeX file. For loading figures
%% from other directories you can use the `import` package
%%   \usepackage{import}
%%
%% and then include the figures with
%%   \import{<path to file>}{<filename>.pgf}
%%
%% Matplotlib used the following preamble
%%   
%%   \makeatletter\@ifpackageloaded{underscore}{}{\usepackage[strings]{underscore}}\makeatother
%%
\begingroup%
\makeatletter%
\begin{pgfpicture}%
\pgfpathrectangle{\pgfpointorigin}{\pgfqpoint{3.264686in}{4.280925in}}%
\pgfusepath{use as bounding box, clip}%
\begin{pgfscope}%
\pgfsetbuttcap%
\pgfsetmiterjoin%
\definecolor{currentfill}{rgb}{1.000000,1.000000,1.000000}%
\pgfsetfillcolor{currentfill}%
\pgfsetlinewidth{0.000000pt}%
\definecolor{currentstroke}{rgb}{1.000000,1.000000,1.000000}%
\pgfsetstrokecolor{currentstroke}%
\pgfsetdash{}{0pt}%
\pgfpathmoveto{\pgfqpoint{-0.000000in}{0.000000in}}%
\pgfpathlineto{\pgfqpoint{3.264686in}{0.000000in}}%
\pgfpathlineto{\pgfqpoint{3.264686in}{4.280925in}}%
\pgfpathlineto{\pgfqpoint{-0.000000in}{4.280925in}}%
\pgfpathlineto{\pgfqpoint{-0.000000in}{0.000000in}}%
\pgfpathclose%
\pgfusepath{fill}%
\end{pgfscope}%
\begin{pgfscope}%
\pgfsetbuttcap%
\pgfsetmiterjoin%
\definecolor{currentfill}{rgb}{1.000000,1.000000,1.000000}%
\pgfsetfillcolor{currentfill}%
\pgfsetlinewidth{0.000000pt}%
\definecolor{currentstroke}{rgb}{0.000000,0.000000,0.000000}%
\pgfsetstrokecolor{currentstroke}%
\pgfsetstrokeopacity{0.000000}%
\pgfsetdash{}{0pt}%
\pgfpathmoveto{\pgfqpoint{0.638581in}{0.499691in}}%
\pgfpathlineto{\pgfqpoint{3.164686in}{0.499691in}}%
\pgfpathlineto{\pgfqpoint{3.164686in}{3.548498in}}%
\pgfpathlineto{\pgfqpoint{0.638581in}{3.548498in}}%
\pgfpathlineto{\pgfqpoint{0.638581in}{0.499691in}}%
\pgfpathclose%
\pgfusepath{fill}%
\end{pgfscope}%
\begin{pgfscope}%
\pgfsetbuttcap%
\pgfsetroundjoin%
\definecolor{currentfill}{rgb}{0.000000,0.000000,0.000000}%
\pgfsetfillcolor{currentfill}%
\pgfsetlinewidth{0.803000pt}%
\definecolor{currentstroke}{rgb}{0.000000,0.000000,0.000000}%
\pgfsetstrokecolor{currentstroke}%
\pgfsetdash{}{0pt}%
\pgfsys@defobject{currentmarker}{\pgfqpoint{0.000000in}{-0.048611in}}{\pgfqpoint{0.000000in}{0.000000in}}{%
\pgfpathmoveto{\pgfqpoint{0.000000in}{0.000000in}}%
\pgfpathlineto{\pgfqpoint{0.000000in}{-0.048611in}}%
\pgfusepath{stroke,fill}%
}%
\begin{pgfscope}%
\pgfsys@transformshift{0.753404in}{0.499691in}%
\pgfsys@useobject{currentmarker}{}%
\end{pgfscope}%
\end{pgfscope}%
\begin{pgfscope}%
\definecolor{textcolor}{rgb}{0.000000,0.000000,0.000000}%
\pgfsetstrokecolor{textcolor}%
\pgfsetfillcolor{textcolor}%
\pgftext[x=0.753404in,y=0.402469in,,top]{\color{textcolor}\rmfamily\fontsize{10.000000}{12.000000}\selectfont \(\displaystyle {1}\)}%
\end{pgfscope}%
\begin{pgfscope}%
\pgfsetbuttcap%
\pgfsetroundjoin%
\definecolor{currentfill}{rgb}{0.000000,0.000000,0.000000}%
\pgfsetfillcolor{currentfill}%
\pgfsetlinewidth{0.803000pt}%
\definecolor{currentstroke}{rgb}{0.000000,0.000000,0.000000}%
\pgfsetstrokecolor{currentstroke}%
\pgfsetdash{}{0pt}%
\pgfsys@defobject{currentmarker}{\pgfqpoint{0.000000in}{-0.048611in}}{\pgfqpoint{0.000000in}{0.000000in}}{%
\pgfpathmoveto{\pgfqpoint{0.000000in}{0.000000in}}%
\pgfpathlineto{\pgfqpoint{0.000000in}{-0.048611in}}%
\pgfusepath{stroke,fill}%
}%
\begin{pgfscope}%
\pgfsys@transformshift{1.040461in}{0.499691in}%
\pgfsys@useobject{currentmarker}{}%
\end{pgfscope}%
\end{pgfscope}%
\begin{pgfscope}%
\definecolor{textcolor}{rgb}{0.000000,0.000000,0.000000}%
\pgfsetstrokecolor{textcolor}%
\pgfsetfillcolor{textcolor}%
\pgftext[x=1.040461in,y=0.402469in,,top]{\color{textcolor}\rmfamily\fontsize{10.000000}{12.000000}\selectfont \(\displaystyle {2}\)}%
\end{pgfscope}%
\begin{pgfscope}%
\pgfsetbuttcap%
\pgfsetroundjoin%
\definecolor{currentfill}{rgb}{0.000000,0.000000,0.000000}%
\pgfsetfillcolor{currentfill}%
\pgfsetlinewidth{0.803000pt}%
\definecolor{currentstroke}{rgb}{0.000000,0.000000,0.000000}%
\pgfsetstrokecolor{currentstroke}%
\pgfsetdash{}{0pt}%
\pgfsys@defobject{currentmarker}{\pgfqpoint{0.000000in}{-0.048611in}}{\pgfqpoint{0.000000in}{0.000000in}}{%
\pgfpathmoveto{\pgfqpoint{0.000000in}{0.000000in}}%
\pgfpathlineto{\pgfqpoint{0.000000in}{-0.048611in}}%
\pgfusepath{stroke,fill}%
}%
\begin{pgfscope}%
\pgfsys@transformshift{1.327519in}{0.499691in}%
\pgfsys@useobject{currentmarker}{}%
\end{pgfscope}%
\end{pgfscope}%
\begin{pgfscope}%
\definecolor{textcolor}{rgb}{0.000000,0.000000,0.000000}%
\pgfsetstrokecolor{textcolor}%
\pgfsetfillcolor{textcolor}%
\pgftext[x=1.327519in,y=0.402469in,,top]{\color{textcolor}\rmfamily\fontsize{10.000000}{12.000000}\selectfont \(\displaystyle {3}\)}%
\end{pgfscope}%
\begin{pgfscope}%
\pgfsetbuttcap%
\pgfsetroundjoin%
\definecolor{currentfill}{rgb}{0.000000,0.000000,0.000000}%
\pgfsetfillcolor{currentfill}%
\pgfsetlinewidth{0.803000pt}%
\definecolor{currentstroke}{rgb}{0.000000,0.000000,0.000000}%
\pgfsetstrokecolor{currentstroke}%
\pgfsetdash{}{0pt}%
\pgfsys@defobject{currentmarker}{\pgfqpoint{0.000000in}{-0.048611in}}{\pgfqpoint{0.000000in}{0.000000in}}{%
\pgfpathmoveto{\pgfqpoint{0.000000in}{0.000000in}}%
\pgfpathlineto{\pgfqpoint{0.000000in}{-0.048611in}}%
\pgfusepath{stroke,fill}%
}%
\begin{pgfscope}%
\pgfsys@transformshift{1.614576in}{0.499691in}%
\pgfsys@useobject{currentmarker}{}%
\end{pgfscope}%
\end{pgfscope}%
\begin{pgfscope}%
\definecolor{textcolor}{rgb}{0.000000,0.000000,0.000000}%
\pgfsetstrokecolor{textcolor}%
\pgfsetfillcolor{textcolor}%
\pgftext[x=1.614576in,y=0.402469in,,top]{\color{textcolor}\rmfamily\fontsize{10.000000}{12.000000}\selectfont \(\displaystyle {4}\)}%
\end{pgfscope}%
\begin{pgfscope}%
\pgfsetbuttcap%
\pgfsetroundjoin%
\definecolor{currentfill}{rgb}{0.000000,0.000000,0.000000}%
\pgfsetfillcolor{currentfill}%
\pgfsetlinewidth{0.803000pt}%
\definecolor{currentstroke}{rgb}{0.000000,0.000000,0.000000}%
\pgfsetstrokecolor{currentstroke}%
\pgfsetdash{}{0pt}%
\pgfsys@defobject{currentmarker}{\pgfqpoint{0.000000in}{-0.048611in}}{\pgfqpoint{0.000000in}{0.000000in}}{%
\pgfpathmoveto{\pgfqpoint{0.000000in}{0.000000in}}%
\pgfpathlineto{\pgfqpoint{0.000000in}{-0.048611in}}%
\pgfusepath{stroke,fill}%
}%
\begin{pgfscope}%
\pgfsys@transformshift{1.901633in}{0.499691in}%
\pgfsys@useobject{currentmarker}{}%
\end{pgfscope}%
\end{pgfscope}%
\begin{pgfscope}%
\definecolor{textcolor}{rgb}{0.000000,0.000000,0.000000}%
\pgfsetstrokecolor{textcolor}%
\pgfsetfillcolor{textcolor}%
\pgftext[x=1.901633in,y=0.402469in,,top]{\color{textcolor}\rmfamily\fontsize{10.000000}{12.000000}\selectfont \(\displaystyle {5}\)}%
\end{pgfscope}%
\begin{pgfscope}%
\pgfsetbuttcap%
\pgfsetroundjoin%
\definecolor{currentfill}{rgb}{0.000000,0.000000,0.000000}%
\pgfsetfillcolor{currentfill}%
\pgfsetlinewidth{0.803000pt}%
\definecolor{currentstroke}{rgb}{0.000000,0.000000,0.000000}%
\pgfsetstrokecolor{currentstroke}%
\pgfsetdash{}{0pt}%
\pgfsys@defobject{currentmarker}{\pgfqpoint{0.000000in}{-0.048611in}}{\pgfqpoint{0.000000in}{0.000000in}}{%
\pgfpathmoveto{\pgfqpoint{0.000000in}{0.000000in}}%
\pgfpathlineto{\pgfqpoint{0.000000in}{-0.048611in}}%
\pgfusepath{stroke,fill}%
}%
\begin{pgfscope}%
\pgfsys@transformshift{2.188691in}{0.499691in}%
\pgfsys@useobject{currentmarker}{}%
\end{pgfscope}%
\end{pgfscope}%
\begin{pgfscope}%
\definecolor{textcolor}{rgb}{0.000000,0.000000,0.000000}%
\pgfsetstrokecolor{textcolor}%
\pgfsetfillcolor{textcolor}%
\pgftext[x=2.188691in,y=0.402469in,,top]{\color{textcolor}\rmfamily\fontsize{10.000000}{12.000000}\selectfont \(\displaystyle {6}\)}%
\end{pgfscope}%
\begin{pgfscope}%
\pgfsetbuttcap%
\pgfsetroundjoin%
\definecolor{currentfill}{rgb}{0.000000,0.000000,0.000000}%
\pgfsetfillcolor{currentfill}%
\pgfsetlinewidth{0.803000pt}%
\definecolor{currentstroke}{rgb}{0.000000,0.000000,0.000000}%
\pgfsetstrokecolor{currentstroke}%
\pgfsetdash{}{0pt}%
\pgfsys@defobject{currentmarker}{\pgfqpoint{0.000000in}{-0.048611in}}{\pgfqpoint{0.000000in}{0.000000in}}{%
\pgfpathmoveto{\pgfqpoint{0.000000in}{0.000000in}}%
\pgfpathlineto{\pgfqpoint{0.000000in}{-0.048611in}}%
\pgfusepath{stroke,fill}%
}%
\begin{pgfscope}%
\pgfsys@transformshift{2.475748in}{0.499691in}%
\pgfsys@useobject{currentmarker}{}%
\end{pgfscope}%
\end{pgfscope}%
\begin{pgfscope}%
\definecolor{textcolor}{rgb}{0.000000,0.000000,0.000000}%
\pgfsetstrokecolor{textcolor}%
\pgfsetfillcolor{textcolor}%
\pgftext[x=2.475748in,y=0.402469in,,top]{\color{textcolor}\rmfamily\fontsize{10.000000}{12.000000}\selectfont \(\displaystyle {7}\)}%
\end{pgfscope}%
\begin{pgfscope}%
\pgfsetbuttcap%
\pgfsetroundjoin%
\definecolor{currentfill}{rgb}{0.000000,0.000000,0.000000}%
\pgfsetfillcolor{currentfill}%
\pgfsetlinewidth{0.803000pt}%
\definecolor{currentstroke}{rgb}{0.000000,0.000000,0.000000}%
\pgfsetstrokecolor{currentstroke}%
\pgfsetdash{}{0pt}%
\pgfsys@defobject{currentmarker}{\pgfqpoint{0.000000in}{-0.048611in}}{\pgfqpoint{0.000000in}{0.000000in}}{%
\pgfpathmoveto{\pgfqpoint{0.000000in}{0.000000in}}%
\pgfpathlineto{\pgfqpoint{0.000000in}{-0.048611in}}%
\pgfusepath{stroke,fill}%
}%
\begin{pgfscope}%
\pgfsys@transformshift{2.762805in}{0.499691in}%
\pgfsys@useobject{currentmarker}{}%
\end{pgfscope}%
\end{pgfscope}%
\begin{pgfscope}%
\definecolor{textcolor}{rgb}{0.000000,0.000000,0.000000}%
\pgfsetstrokecolor{textcolor}%
\pgfsetfillcolor{textcolor}%
\pgftext[x=2.762805in,y=0.402469in,,top]{\color{textcolor}\rmfamily\fontsize{10.000000}{12.000000}\selectfont \(\displaystyle {8}\)}%
\end{pgfscope}%
\begin{pgfscope}%
\pgfsetbuttcap%
\pgfsetroundjoin%
\definecolor{currentfill}{rgb}{0.000000,0.000000,0.000000}%
\pgfsetfillcolor{currentfill}%
\pgfsetlinewidth{0.803000pt}%
\definecolor{currentstroke}{rgb}{0.000000,0.000000,0.000000}%
\pgfsetstrokecolor{currentstroke}%
\pgfsetdash{}{0pt}%
\pgfsys@defobject{currentmarker}{\pgfqpoint{0.000000in}{-0.048611in}}{\pgfqpoint{0.000000in}{0.000000in}}{%
\pgfpathmoveto{\pgfqpoint{0.000000in}{0.000000in}}%
\pgfpathlineto{\pgfqpoint{0.000000in}{-0.048611in}}%
\pgfusepath{stroke,fill}%
}%
\begin{pgfscope}%
\pgfsys@transformshift{3.049863in}{0.499691in}%
\pgfsys@useobject{currentmarker}{}%
\end{pgfscope}%
\end{pgfscope}%
\begin{pgfscope}%
\definecolor{textcolor}{rgb}{0.000000,0.000000,0.000000}%
\pgfsetstrokecolor{textcolor}%
\pgfsetfillcolor{textcolor}%
\pgftext[x=3.049863in,y=0.402469in,,top]{\color{textcolor}\rmfamily\fontsize{10.000000}{12.000000}\selectfont \(\displaystyle {9}\)}%
\end{pgfscope}%
\begin{pgfscope}%
\definecolor{textcolor}{rgb}{0.000000,0.000000,0.000000}%
\pgfsetstrokecolor{textcolor}%
\pgfsetfillcolor{textcolor}%
\pgftext[x=1.901633in,y=0.223457in,,top]{\color{textcolor}\rmfamily\fontsize{10.000000}{12.000000}\selectfont gzip compression level}%
\end{pgfscope}%
\begin{pgfscope}%
\pgfsetbuttcap%
\pgfsetroundjoin%
\definecolor{currentfill}{rgb}{0.000000,0.000000,0.000000}%
\pgfsetfillcolor{currentfill}%
\pgfsetlinewidth{0.803000pt}%
\definecolor{currentstroke}{rgb}{0.000000,0.000000,0.000000}%
\pgfsetstrokecolor{currentstroke}%
\pgfsetdash{}{0pt}%
\pgfsys@defobject{currentmarker}{\pgfqpoint{-0.048611in}{0.000000in}}{\pgfqpoint{-0.000000in}{0.000000in}}{%
\pgfpathmoveto{\pgfqpoint{-0.000000in}{0.000000in}}%
\pgfpathlineto{\pgfqpoint{-0.048611in}{0.000000in}}%
\pgfusepath{stroke,fill}%
}%
\begin{pgfscope}%
\pgfsys@transformshift{0.638581in}{0.974230in}%
\pgfsys@useobject{currentmarker}{}%
\end{pgfscope}%
\end{pgfscope}%
\begin{pgfscope}%
\definecolor{textcolor}{rgb}{0.000000,0.000000,0.000000}%
\pgfsetstrokecolor{textcolor}%
\pgfsetfillcolor{textcolor}%
\pgftext[x=0.294444in, y=0.926005in, left, base]{\color{textcolor}\rmfamily\fontsize{10.000000}{12.000000}\selectfont \(\displaystyle {0.70}\)}%
\end{pgfscope}%
\begin{pgfscope}%
\pgfsetbuttcap%
\pgfsetroundjoin%
\definecolor{currentfill}{rgb}{0.000000,0.000000,0.000000}%
\pgfsetfillcolor{currentfill}%
\pgfsetlinewidth{0.803000pt}%
\definecolor{currentstroke}{rgb}{0.000000,0.000000,0.000000}%
\pgfsetstrokecolor{currentstroke}%
\pgfsetdash{}{0pt}%
\pgfsys@defobject{currentmarker}{\pgfqpoint{-0.048611in}{0.000000in}}{\pgfqpoint{-0.000000in}{0.000000in}}{%
\pgfpathmoveto{\pgfqpoint{-0.000000in}{0.000000in}}%
\pgfpathlineto{\pgfqpoint{-0.048611in}{0.000000in}}%
\pgfusepath{stroke,fill}%
}%
\begin{pgfscope}%
\pgfsys@transformshift{0.638581in}{1.499162in}%
\pgfsys@useobject{currentmarker}{}%
\end{pgfscope}%
\end{pgfscope}%
\begin{pgfscope}%
\definecolor{textcolor}{rgb}{0.000000,0.000000,0.000000}%
\pgfsetstrokecolor{textcolor}%
\pgfsetfillcolor{textcolor}%
\pgftext[x=0.294444in, y=1.450937in, left, base]{\color{textcolor}\rmfamily\fontsize{10.000000}{12.000000}\selectfont \(\displaystyle {0.75}\)}%
\end{pgfscope}%
\begin{pgfscope}%
\pgfsetbuttcap%
\pgfsetroundjoin%
\definecolor{currentfill}{rgb}{0.000000,0.000000,0.000000}%
\pgfsetfillcolor{currentfill}%
\pgfsetlinewidth{0.803000pt}%
\definecolor{currentstroke}{rgb}{0.000000,0.000000,0.000000}%
\pgfsetstrokecolor{currentstroke}%
\pgfsetdash{}{0pt}%
\pgfsys@defobject{currentmarker}{\pgfqpoint{-0.048611in}{0.000000in}}{\pgfqpoint{-0.000000in}{0.000000in}}{%
\pgfpathmoveto{\pgfqpoint{-0.000000in}{0.000000in}}%
\pgfpathlineto{\pgfqpoint{-0.048611in}{0.000000in}}%
\pgfusepath{stroke,fill}%
}%
\begin{pgfscope}%
\pgfsys@transformshift{0.638581in}{2.024095in}%
\pgfsys@useobject{currentmarker}{}%
\end{pgfscope}%
\end{pgfscope}%
\begin{pgfscope}%
\definecolor{textcolor}{rgb}{0.000000,0.000000,0.000000}%
\pgfsetstrokecolor{textcolor}%
\pgfsetfillcolor{textcolor}%
\pgftext[x=0.294444in, y=1.975869in, left, base]{\color{textcolor}\rmfamily\fontsize{10.000000}{12.000000}\selectfont \(\displaystyle {0.80}\)}%
\end{pgfscope}%
\begin{pgfscope}%
\pgfsetbuttcap%
\pgfsetroundjoin%
\definecolor{currentfill}{rgb}{0.000000,0.000000,0.000000}%
\pgfsetfillcolor{currentfill}%
\pgfsetlinewidth{0.803000pt}%
\definecolor{currentstroke}{rgb}{0.000000,0.000000,0.000000}%
\pgfsetstrokecolor{currentstroke}%
\pgfsetdash{}{0pt}%
\pgfsys@defobject{currentmarker}{\pgfqpoint{-0.048611in}{0.000000in}}{\pgfqpoint{-0.000000in}{0.000000in}}{%
\pgfpathmoveto{\pgfqpoint{-0.000000in}{0.000000in}}%
\pgfpathlineto{\pgfqpoint{-0.048611in}{0.000000in}}%
\pgfusepath{stroke,fill}%
}%
\begin{pgfscope}%
\pgfsys@transformshift{0.638581in}{2.549027in}%
\pgfsys@useobject{currentmarker}{}%
\end{pgfscope}%
\end{pgfscope}%
\begin{pgfscope}%
\definecolor{textcolor}{rgb}{0.000000,0.000000,0.000000}%
\pgfsetstrokecolor{textcolor}%
\pgfsetfillcolor{textcolor}%
\pgftext[x=0.294444in, y=2.500802in, left, base]{\color{textcolor}\rmfamily\fontsize{10.000000}{12.000000}\selectfont \(\displaystyle {0.85}\)}%
\end{pgfscope}%
\begin{pgfscope}%
\pgfsetbuttcap%
\pgfsetroundjoin%
\definecolor{currentfill}{rgb}{0.000000,0.000000,0.000000}%
\pgfsetfillcolor{currentfill}%
\pgfsetlinewidth{0.803000pt}%
\definecolor{currentstroke}{rgb}{0.000000,0.000000,0.000000}%
\pgfsetstrokecolor{currentstroke}%
\pgfsetdash{}{0pt}%
\pgfsys@defobject{currentmarker}{\pgfqpoint{-0.048611in}{0.000000in}}{\pgfqpoint{-0.000000in}{0.000000in}}{%
\pgfpathmoveto{\pgfqpoint{-0.000000in}{0.000000in}}%
\pgfpathlineto{\pgfqpoint{-0.048611in}{0.000000in}}%
\pgfusepath{stroke,fill}%
}%
\begin{pgfscope}%
\pgfsys@transformshift{0.638581in}{3.073960in}%
\pgfsys@useobject{currentmarker}{}%
\end{pgfscope}%
\end{pgfscope}%
\begin{pgfscope}%
\definecolor{textcolor}{rgb}{0.000000,0.000000,0.000000}%
\pgfsetstrokecolor{textcolor}%
\pgfsetfillcolor{textcolor}%
\pgftext[x=0.294444in, y=3.025734in, left, base]{\color{textcolor}\rmfamily\fontsize{10.000000}{12.000000}\selectfont \(\displaystyle {0.90}\)}%
\end{pgfscope}%
\begin{pgfscope}%
\definecolor{textcolor}{rgb}{0.000000,0.000000,0.000000}%
\pgfsetstrokecolor{textcolor}%
\pgfsetfillcolor{textcolor}%
\pgftext[x=0.238889in,y=2.024095in,,bottom,rotate=90.000000]{\color{textcolor}\rmfamily\fontsize{10.000000}{12.000000}\selectfont decompression time (s)}%
\end{pgfscope}%
\begin{pgfscope}%
\pgfpathrectangle{\pgfqpoint{0.638581in}{0.499691in}}{\pgfqpoint{2.526105in}{3.048807in}}%
\pgfusepath{clip}%
\pgfsetrectcap%
\pgfsetroundjoin%
\pgfsetlinewidth{1.505625pt}%
\definecolor{currentstroke}{rgb}{0.090196,0.745098,0.811765}%
\pgfsetstrokecolor{currentstroke}%
\pgfsetdash{}{0pt}%
\pgfpathmoveto{\pgfqpoint{0.753404in}{3.325927in}}%
\pgfpathlineto{\pgfqpoint{1.040461in}{3.220941in}}%
\pgfpathlineto{\pgfqpoint{1.327519in}{2.779997in}}%
\pgfpathlineto{\pgfqpoint{1.614576in}{2.821992in}}%
\pgfpathlineto{\pgfqpoint{1.901633in}{2.654014in}}%
\pgfpathlineto{\pgfqpoint{2.188691in}{2.360051in}}%
\pgfpathlineto{\pgfqpoint{2.475748in}{2.507033in}}%
\pgfpathlineto{\pgfqpoint{2.762805in}{2.486035in}}%
\pgfpathlineto{\pgfqpoint{3.049863in}{2.381049in}}%
\pgfusepath{stroke}%
\end{pgfscope}%
\begin{pgfscope}%
\pgfpathrectangle{\pgfqpoint{0.638581in}{0.499691in}}{\pgfqpoint{2.526105in}{3.048807in}}%
\pgfusepath{clip}%
\pgfsetrectcap%
\pgfsetroundjoin%
\pgfsetlinewidth{1.505625pt}%
\definecolor{currentstroke}{rgb}{0.737255,0.741176,0.133333}%
\pgfsetstrokecolor{currentstroke}%
\pgfsetdash{}{0pt}%
\pgfpathmoveto{\pgfqpoint{0.753404in}{3.010968in}}%
\pgfpathlineto{\pgfqpoint{1.040461in}{3.283932in}}%
\pgfpathlineto{\pgfqpoint{1.327519in}{3.409916in}}%
\pgfpathlineto{\pgfqpoint{1.614576in}{2.402046in}}%
\pgfpathlineto{\pgfqpoint{1.901633in}{2.381049in}}%
\pgfpathlineto{\pgfqpoint{2.188691in}{2.297060in}}%
\pgfpathlineto{\pgfqpoint{2.475748in}{2.360051in}}%
\pgfpathlineto{\pgfqpoint{2.762805in}{2.234068in}}%
\pgfpathlineto{\pgfqpoint{3.049863in}{2.108084in}}%
\pgfusepath{stroke}%
\end{pgfscope}%
\begin{pgfscope}%
\pgfpathrectangle{\pgfqpoint{0.638581in}{0.499691in}}{\pgfqpoint{2.526105in}{3.048807in}}%
\pgfusepath{clip}%
\pgfsetrectcap%
\pgfsetroundjoin%
\pgfsetlinewidth{1.505625pt}%
\definecolor{currentstroke}{rgb}{0.839216,0.152941,0.156863}%
\pgfsetstrokecolor{currentstroke}%
\pgfsetdash{}{0pt}%
\pgfpathmoveto{\pgfqpoint{0.753404in}{1.457168in}}%
\pgfpathlineto{\pgfqpoint{1.040461in}{1.709135in}}%
\pgfpathlineto{\pgfqpoint{1.327519in}{2.066089in}}%
\pgfpathlineto{\pgfqpoint{1.614576in}{2.360051in}}%
\pgfpathlineto{\pgfqpoint{1.901633in}{2.276062in}}%
\pgfpathlineto{\pgfqpoint{2.188691in}{2.192073in}}%
\pgfpathlineto{\pgfqpoint{2.475748in}{2.612019in}}%
\pgfpathlineto{\pgfqpoint{2.762805in}{2.234068in}}%
\pgfpathlineto{\pgfqpoint{3.049863in}{2.255065in}}%
\pgfusepath{stroke}%
\end{pgfscope}%
\begin{pgfscope}%
\pgfpathrectangle{\pgfqpoint{0.638581in}{0.499691in}}{\pgfqpoint{2.526105in}{3.048807in}}%
\pgfusepath{clip}%
\pgfsetrectcap%
\pgfsetroundjoin%
\pgfsetlinewidth{1.505625pt}%
\definecolor{currentstroke}{rgb}{0.890196,0.466667,0.760784}%
\pgfsetstrokecolor{currentstroke}%
\pgfsetdash{}{0pt}%
\pgfpathmoveto{\pgfqpoint{0.753404in}{2.444041in}}%
\pgfpathlineto{\pgfqpoint{1.040461in}{2.339054in}}%
\pgfpathlineto{\pgfqpoint{1.327519in}{1.520160in}}%
\pgfpathlineto{\pgfqpoint{1.614576in}{1.541157in}}%
\pgfpathlineto{\pgfqpoint{1.901633in}{1.604149in}}%
\pgfpathlineto{\pgfqpoint{2.188691in}{1.604149in}}%
\pgfpathlineto{\pgfqpoint{2.475748in}{1.520160in}}%
\pgfpathlineto{\pgfqpoint{2.762805in}{1.499162in}}%
\pgfpathlineto{\pgfqpoint{3.049863in}{1.625146in}}%
\pgfusepath{stroke}%
\end{pgfscope}%
\begin{pgfscope}%
\pgfpathrectangle{\pgfqpoint{0.638581in}{0.499691in}}{\pgfqpoint{2.526105in}{3.048807in}}%
\pgfusepath{clip}%
\pgfsetrectcap%
\pgfsetroundjoin%
\pgfsetlinewidth{1.505625pt}%
\definecolor{currentstroke}{rgb}{0.498039,0.498039,0.498039}%
\pgfsetstrokecolor{currentstroke}%
\pgfsetdash{}{0pt}%
\pgfpathmoveto{\pgfqpoint{0.753404in}{2.423043in}}%
\pgfpathlineto{\pgfqpoint{1.040461in}{2.234068in}}%
\pgfpathlineto{\pgfqpoint{1.327519in}{1.226198in}}%
\pgfpathlineto{\pgfqpoint{1.614576in}{1.394176in}}%
\pgfpathlineto{\pgfqpoint{1.901633in}{1.478165in}}%
\pgfpathlineto{\pgfqpoint{2.188691in}{1.499162in}}%
\pgfpathlineto{\pgfqpoint{2.475748in}{1.205200in}}%
\pgfpathlineto{\pgfqpoint{2.762805in}{1.226198in}}%
\pgfpathlineto{\pgfqpoint{3.049863in}{1.289189in}}%
\pgfusepath{stroke}%
\end{pgfscope}%
\begin{pgfscope}%
\pgfpathrectangle{\pgfqpoint{0.638581in}{0.499691in}}{\pgfqpoint{2.526105in}{3.048807in}}%
\pgfusepath{clip}%
\pgfsetbuttcap%
\pgfsetroundjoin%
\pgfsetlinewidth{1.505625pt}%
\definecolor{currentstroke}{rgb}{0.498039,0.498039,0.498039}%
\pgfsetstrokecolor{currentstroke}%
\pgfsetdash{{5.550000pt}{2.400000pt}}{0.000000pt}%
\pgfpathmoveto{\pgfqpoint{0.753404in}{2.192073in}}%
\pgfpathlineto{\pgfqpoint{1.040461in}{2.192073in}}%
\pgfpathlineto{\pgfqpoint{1.327519in}{0.785254in}}%
\pgfpathlineto{\pgfqpoint{1.614576in}{0.995227in}}%
\pgfpathlineto{\pgfqpoint{1.901633in}{1.352181in}}%
\pgfpathlineto{\pgfqpoint{2.188691in}{1.310187in}}%
\pgfpathlineto{\pgfqpoint{2.475748in}{1.226198in}}%
\pgfpathlineto{\pgfqpoint{2.762805in}{1.352181in}}%
\pgfpathlineto{\pgfqpoint{3.049863in}{1.625146in}}%
\pgfusepath{stroke}%
\end{pgfscope}%
\begin{pgfscope}%
\pgfpathrectangle{\pgfqpoint{0.638581in}{0.499691in}}{\pgfqpoint{2.526105in}{3.048807in}}%
\pgfusepath{clip}%
\pgfsetbuttcap%
\pgfsetroundjoin%
\pgfsetlinewidth{1.505625pt}%
\definecolor{currentstroke}{rgb}{0.121569,0.466667,0.705882}%
\pgfsetstrokecolor{currentstroke}%
\pgfsetdash{{9.600000pt}{2.400000pt}{1.500000pt}{2.400000pt}}{0.000000pt}%
\pgfpathmoveto{\pgfqpoint{0.753404in}{2.339054in}}%
\pgfpathlineto{\pgfqpoint{1.040461in}{2.045092in}}%
\pgfpathlineto{\pgfqpoint{1.327519in}{1.226198in}}%
\pgfpathlineto{\pgfqpoint{1.614576in}{1.268192in}}%
\pgfpathlineto{\pgfqpoint{1.901633in}{1.226198in}}%
\pgfpathlineto{\pgfqpoint{2.188691in}{1.247195in}}%
\pgfpathlineto{\pgfqpoint{2.475748in}{1.226198in}}%
\pgfpathlineto{\pgfqpoint{2.762805in}{1.289189in}}%
\pgfpathlineto{\pgfqpoint{3.049863in}{1.268192in}}%
\pgfusepath{stroke}%
\end{pgfscope}%
\begin{pgfscope}%
\pgfpathrectangle{\pgfqpoint{0.638581in}{0.499691in}}{\pgfqpoint{2.526105in}{3.048807in}}%
\pgfusepath{clip}%
\pgfsetbuttcap%
\pgfsetroundjoin%
\pgfsetlinewidth{1.505625pt}%
\definecolor{currentstroke}{rgb}{0.890196,0.466667,0.760784}%
\pgfsetstrokecolor{currentstroke}%
\pgfsetdash{{1.500000pt}{2.475000pt}}{0.000000pt}%
\pgfpathmoveto{\pgfqpoint{0.753404in}{2.171076in}}%
\pgfpathlineto{\pgfqpoint{1.040461in}{2.066089in}}%
\pgfpathlineto{\pgfqpoint{1.327519in}{0.995227in}}%
\pgfpathlineto{\pgfqpoint{1.614576in}{1.100214in}}%
\pgfpathlineto{\pgfqpoint{1.901633in}{1.331184in}}%
\pgfpathlineto{\pgfqpoint{2.188691in}{1.226198in}}%
\pgfpathlineto{\pgfqpoint{2.475748in}{1.205200in}}%
\pgfpathlineto{\pgfqpoint{2.762805in}{1.226198in}}%
\pgfpathlineto{\pgfqpoint{3.049863in}{1.205200in}}%
\pgfusepath{stroke}%
\end{pgfscope}%
\begin{pgfscope}%
\pgfpathrectangle{\pgfqpoint{0.638581in}{0.499691in}}{\pgfqpoint{2.526105in}{3.048807in}}%
\pgfusepath{clip}%
\pgfsetbuttcap%
\pgfsetroundjoin%
\pgfsetlinewidth{1.505625pt}%
\definecolor{currentstroke}{rgb}{0.498039,0.498039,0.498039}%
\pgfsetstrokecolor{currentstroke}%
\pgfsetdash{{9.600000pt}{2.400000pt}{1.500000pt}{2.400000pt}}{0.000000pt}%
\pgfpathmoveto{\pgfqpoint{0.753404in}{2.465038in}}%
\pgfpathlineto{\pgfqpoint{1.040461in}{2.171076in}}%
\pgfpathlineto{\pgfqpoint{1.327519in}{0.806252in}}%
\pgfpathlineto{\pgfqpoint{1.614576in}{1.163206in}}%
\pgfpathlineto{\pgfqpoint{1.901633in}{0.932235in}}%
\pgfpathlineto{\pgfqpoint{2.188691in}{1.016225in}}%
\pgfpathlineto{\pgfqpoint{2.475748in}{1.037222in}}%
\pgfpathlineto{\pgfqpoint{2.762805in}{0.911238in}}%
\pgfpathlineto{\pgfqpoint{3.049863in}{1.037222in}}%
\pgfusepath{stroke}%
\end{pgfscope}%
\begin{pgfscope}%
\pgfpathrectangle{\pgfqpoint{0.638581in}{0.499691in}}{\pgfqpoint{2.526105in}{3.048807in}}%
\pgfusepath{clip}%
\pgfsetbuttcap%
\pgfsetroundjoin%
\pgfsetlinewidth{1.505625pt}%
\definecolor{currentstroke}{rgb}{0.498039,0.498039,0.498039}%
\pgfsetstrokecolor{currentstroke}%
\pgfsetdash{{1.500000pt}{2.475000pt}}{0.000000pt}%
\pgfpathmoveto{\pgfqpoint{0.753404in}{2.318057in}}%
\pgfpathlineto{\pgfqpoint{1.040461in}{1.982100in}}%
\pgfpathlineto{\pgfqpoint{1.327519in}{0.701265in}}%
\pgfpathlineto{\pgfqpoint{1.614576in}{0.974230in}}%
\pgfpathlineto{\pgfqpoint{1.901633in}{0.995227in}}%
\pgfpathlineto{\pgfqpoint{2.188691in}{0.995227in}}%
\pgfpathlineto{\pgfqpoint{2.475748in}{1.142208in}}%
\pgfpathlineto{\pgfqpoint{2.762805in}{1.142208in}}%
\pgfpathlineto{\pgfqpoint{3.049863in}{1.016225in}}%
\pgfusepath{stroke}%
\end{pgfscope}%
\begin{pgfscope}%
\pgfpathrectangle{\pgfqpoint{0.638581in}{0.499691in}}{\pgfqpoint{2.526105in}{3.048807in}}%
\pgfusepath{clip}%
\pgfsetbuttcap%
\pgfsetroundjoin%
\pgfsetlinewidth{1.505625pt}%
\definecolor{currentstroke}{rgb}{0.890196,0.466667,0.760784}%
\pgfsetstrokecolor{currentstroke}%
\pgfsetdash{{5.550000pt}{2.400000pt}}{0.000000pt}%
\pgfpathmoveto{\pgfqpoint{0.753404in}{2.150079in}}%
\pgfpathlineto{\pgfqpoint{1.040461in}{2.087087in}}%
\pgfpathlineto{\pgfqpoint{1.327519in}{0.806252in}}%
\pgfpathlineto{\pgfqpoint{1.614576in}{0.932235in}}%
\pgfpathlineto{\pgfqpoint{1.901633in}{0.995227in}}%
\pgfpathlineto{\pgfqpoint{2.188691in}{0.974230in}}%
\pgfpathlineto{\pgfqpoint{2.475748in}{0.974230in}}%
\pgfpathlineto{\pgfqpoint{2.762805in}{0.911238in}}%
\pgfpathlineto{\pgfqpoint{3.049863in}{0.974230in}}%
\pgfusepath{stroke}%
\end{pgfscope}%
\begin{pgfscope}%
\pgfpathrectangle{\pgfqpoint{0.638581in}{0.499691in}}{\pgfqpoint{2.526105in}{3.048807in}}%
\pgfusepath{clip}%
\pgfsetbuttcap%
\pgfsetroundjoin%
\pgfsetlinewidth{1.505625pt}%
\definecolor{currentstroke}{rgb}{0.890196,0.466667,0.760784}%
\pgfsetstrokecolor{currentstroke}%
\pgfsetdash{{9.600000pt}{2.400000pt}{1.500000pt}{2.400000pt}}{0.000000pt}%
\pgfpathmoveto{\pgfqpoint{0.753404in}{2.024095in}}%
\pgfpathlineto{\pgfqpoint{1.040461in}{2.150079in}}%
\pgfpathlineto{\pgfqpoint{1.327519in}{0.722262in}}%
\pgfpathlineto{\pgfqpoint{1.614576in}{0.890241in}}%
\pgfpathlineto{\pgfqpoint{1.901633in}{0.974230in}}%
\pgfpathlineto{\pgfqpoint{2.188691in}{0.869244in}}%
\pgfpathlineto{\pgfqpoint{2.475748in}{0.848246in}}%
\pgfpathlineto{\pgfqpoint{2.762805in}{0.869244in}}%
\pgfpathlineto{\pgfqpoint{3.049863in}{0.890241in}}%
\pgfusepath{stroke}%
\end{pgfscope}%
\begin{pgfscope}%
\pgfpathrectangle{\pgfqpoint{0.638581in}{0.499691in}}{\pgfqpoint{2.526105in}{3.048807in}}%
\pgfusepath{clip}%
\pgfsetrectcap%
\pgfsetroundjoin%
\pgfsetlinewidth{1.505625pt}%
\definecolor{currentstroke}{rgb}{0.172549,0.627451,0.172549}%
\pgfsetstrokecolor{currentstroke}%
\pgfsetdash{}{0pt}%
\pgfpathmoveto{\pgfqpoint{0.753404in}{2.171076in}}%
\pgfpathlineto{\pgfqpoint{1.040461in}{1.814122in}}%
\pgfpathlineto{\pgfqpoint{1.327519in}{0.869244in}}%
\pgfpathlineto{\pgfqpoint{1.614576in}{1.100214in}}%
\pgfpathlineto{\pgfqpoint{1.901633in}{1.184203in}}%
\pgfpathlineto{\pgfqpoint{2.188691in}{0.659271in}}%
\pgfpathlineto{\pgfqpoint{2.475748in}{0.638273in}}%
\pgfpathlineto{\pgfqpoint{2.762805in}{0.638273in}}%
\pgfpathlineto{\pgfqpoint{3.049863in}{0.743260in}}%
\pgfusepath{stroke}%
\end{pgfscope}%
\begin{pgfscope}%
\pgfpathrectangle{\pgfqpoint{0.638581in}{0.499691in}}{\pgfqpoint{2.526105in}{3.048807in}}%
\pgfusepath{clip}%
\pgfsetbuttcap%
\pgfsetroundjoin%
\pgfsetlinewidth{1.505625pt}%
\definecolor{currentstroke}{rgb}{0.000000,0.000000,0.000000}%
\pgfsetstrokecolor{currentstroke}%
\pgfsetstrokeopacity{0.500000}%
\pgfsetdash{{5.550000pt}{2.400000pt}}{0.000000pt}%
\pgfpathmoveto{\pgfqpoint{2.188691in}{0.499691in}}%
\pgfpathlineto{\pgfqpoint{2.188691in}{3.548498in}}%
\pgfusepath{stroke}%
\end{pgfscope}%
\begin{pgfscope}%
\pgfsetrectcap%
\pgfsetmiterjoin%
\pgfsetlinewidth{0.803000pt}%
\definecolor{currentstroke}{rgb}{0.000000,0.000000,0.000000}%
\pgfsetstrokecolor{currentstroke}%
\pgfsetdash{}{0pt}%
\pgfpathmoveto{\pgfqpoint{0.638581in}{0.499691in}}%
\pgfpathlineto{\pgfqpoint{0.638581in}{3.548498in}}%
\pgfusepath{stroke}%
\end{pgfscope}%
\begin{pgfscope}%
\pgfsetrectcap%
\pgfsetmiterjoin%
\pgfsetlinewidth{0.803000pt}%
\definecolor{currentstroke}{rgb}{0.000000,0.000000,0.000000}%
\pgfsetstrokecolor{currentstroke}%
\pgfsetdash{}{0pt}%
\pgfpathmoveto{\pgfqpoint{3.164686in}{0.499691in}}%
\pgfpathlineto{\pgfqpoint{3.164686in}{3.548498in}}%
\pgfusepath{stroke}%
\end{pgfscope}%
\begin{pgfscope}%
\pgfsetrectcap%
\pgfsetmiterjoin%
\pgfsetlinewidth{0.803000pt}%
\definecolor{currentstroke}{rgb}{0.000000,0.000000,0.000000}%
\pgfsetstrokecolor{currentstroke}%
\pgfsetdash{}{0pt}%
\pgfpathmoveto{\pgfqpoint{0.638581in}{0.499691in}}%
\pgfpathlineto{\pgfqpoint{3.164686in}{0.499691in}}%
\pgfusepath{stroke}%
\end{pgfscope}%
\begin{pgfscope}%
\pgfsetrectcap%
\pgfsetmiterjoin%
\pgfsetlinewidth{0.803000pt}%
\definecolor{currentstroke}{rgb}{0.000000,0.000000,0.000000}%
\pgfsetstrokecolor{currentstroke}%
\pgfsetdash{}{0pt}%
\pgfpathmoveto{\pgfqpoint{0.638581in}{3.548498in}}%
\pgfpathlineto{\pgfqpoint{3.164686in}{3.548498in}}%
\pgfusepath{stroke}%
\end{pgfscope}%
\begin{pgfscope}%
\pgfsetbuttcap%
\pgfsetmiterjoin%
\definecolor{currentfill}{rgb}{1.000000,1.000000,1.000000}%
\pgfsetfillcolor{currentfill}%
\pgfsetfillopacity{0.800000}%
\pgfsetlinewidth{1.003750pt}%
\definecolor{currentstroke}{rgb}{0.800000,0.800000,0.800000}%
\pgfsetstrokecolor{currentstroke}%
\pgfsetstrokeopacity{0.800000}%
\pgfsetdash{}{0pt}%
\pgfpathmoveto{\pgfqpoint{0.658025in}{3.609475in}}%
\pgfpathlineto{\pgfqpoint{3.145241in}{3.609475in}}%
\pgfpathquadraticcurveto{\pgfqpoint{3.164686in}{3.609475in}}{\pgfqpoint{3.164686in}{3.628919in}}%
\pgfpathlineto{\pgfqpoint{3.164686in}{4.161480in}}%
\pgfpathquadraticcurveto{\pgfqpoint{3.164686in}{4.180925in}}{\pgfqpoint{3.145241in}{4.180925in}}%
\pgfpathlineto{\pgfqpoint{0.658025in}{4.180925in}}%
\pgfpathquadraticcurveto{\pgfqpoint{0.638581in}{4.180925in}}{\pgfqpoint{0.638581in}{4.161480in}}%
\pgfpathlineto{\pgfqpoint{0.638581in}{3.628919in}}%
\pgfpathquadraticcurveto{\pgfqpoint{0.638581in}{3.609475in}}{\pgfqpoint{0.658025in}{3.609475in}}%
\pgfpathlineto{\pgfqpoint{0.658025in}{3.609475in}}%
\pgfpathclose%
\pgfusepath{stroke,fill}%
\end{pgfscope}%
\begin{pgfscope}%
\pgfsetrectcap%
\pgfsetroundjoin%
\pgfsetlinewidth{1.505625pt}%
\definecolor{currentstroke}{rgb}{0.090196,0.745098,0.811765}%
\pgfsetstrokecolor{currentstroke}%
\pgfsetdash{}{0pt}%
\pgfpathmoveto{\pgfqpoint{0.677470in}{4.108008in}}%
\pgfpathlineto{\pgfqpoint{0.774692in}{4.108008in}}%
\pgfpathlineto{\pgfqpoint{0.871914in}{4.108008in}}%
\pgfusepath{stroke}%
\end{pgfscope}%
\begin{pgfscope}%
\definecolor{textcolor}{rgb}{0.000000,0.000000,0.000000}%
\pgfsetstrokecolor{textcolor}%
\pgfsetfillcolor{textcolor}%
\pgftext[x=0.949692in,y=4.073980in,left,base]{\color{textcolor}\rmfamily\fontsize{7.000000}{8.400000}\selectfont c}%
\end{pgfscope}%
\begin{pgfscope}%
\pgfsetrectcap%
\pgfsetroundjoin%
\pgfsetlinewidth{1.505625pt}%
\definecolor{currentstroke}{rgb}{0.737255,0.741176,0.133333}%
\pgfsetstrokecolor{currentstroke}%
\pgfsetdash{}{0pt}%
\pgfpathmoveto{\pgfqpoint{0.677470in}{3.972437in}}%
\pgfpathlineto{\pgfqpoint{0.774692in}{3.972437in}}%
\pgfpathlineto{\pgfqpoint{0.871914in}{3.972437in}}%
\pgfusepath{stroke}%
\end{pgfscope}%
\begin{pgfscope}%
\definecolor{textcolor}{rgb}{0.000000,0.000000,0.000000}%
\pgfsetstrokecolor{textcolor}%
\pgfsetfillcolor{textcolor}%
\pgftext[x=0.949692in,y=3.938409in,left,base]{\color{textcolor}\rmfamily\fontsize{7.000000}{8.400000}\selectfont a}%
\end{pgfscope}%
\begin{pgfscope}%
\pgfsetrectcap%
\pgfsetroundjoin%
\pgfsetlinewidth{1.505625pt}%
\definecolor{currentstroke}{rgb}{0.839216,0.152941,0.156863}%
\pgfsetstrokecolor{currentstroke}%
\pgfsetdash{}{0pt}%
\pgfpathmoveto{\pgfqpoint{0.677470in}{3.836866in}}%
\pgfpathlineto{\pgfqpoint{0.774692in}{3.836866in}}%
\pgfpathlineto{\pgfqpoint{0.871914in}{3.836866in}}%
\pgfusepath{stroke}%
\end{pgfscope}%
\begin{pgfscope}%
\definecolor{textcolor}{rgb}{0.000000,0.000000,0.000000}%
\pgfsetstrokecolor{textcolor}%
\pgfsetfillcolor{textcolor}%
\pgftext[x=0.949692in,y=3.802838in,left,base]{\color{textcolor}\rmfamily\fontsize{7.000000}{8.400000}\selectfont x}%
\end{pgfscope}%
\begin{pgfscope}%
\pgfsetrectcap%
\pgfsetroundjoin%
\pgfsetlinewidth{1.505625pt}%
\definecolor{currentstroke}{rgb}{0.890196,0.466667,0.760784}%
\pgfsetstrokecolor{currentstroke}%
\pgfsetdash{}{0pt}%
\pgfpathmoveto{\pgfqpoint{0.677470in}{3.701295in}}%
\pgfpathlineto{\pgfqpoint{0.774692in}{3.701295in}}%
\pgfpathlineto{\pgfqpoint{0.871914in}{3.701295in}}%
\pgfusepath{stroke}%
\end{pgfscope}%
\begin{pgfscope}%
\definecolor{textcolor}{rgb}{0.000000,0.000000,0.000000}%
\pgfsetstrokecolor{textcolor}%
\pgfsetfillcolor{textcolor}%
\pgftext[x=0.949692in,y=3.667268in,left,base]{\color{textcolor}\rmfamily\fontsize{7.000000}{8.400000}\selectfont d1}%
\end{pgfscope}%
\begin{pgfscope}%
\pgfsetrectcap%
\pgfsetroundjoin%
\pgfsetlinewidth{1.505625pt}%
\definecolor{currentstroke}{rgb}{0.498039,0.498039,0.498039}%
\pgfsetstrokecolor{currentstroke}%
\pgfsetdash{}{0pt}%
\pgfpathmoveto{\pgfqpoint{1.363936in}{4.108008in}}%
\pgfpathlineto{\pgfqpoint{1.461158in}{4.108008in}}%
\pgfpathlineto{\pgfqpoint{1.558381in}{4.108008in}}%
\pgfusepath{stroke}%
\end{pgfscope}%
\begin{pgfscope}%
\definecolor{textcolor}{rgb}{0.000000,0.000000,0.000000}%
\pgfsetstrokecolor{textcolor}%
\pgfsetfillcolor{textcolor}%
\pgftext[x=1.636158in,y=4.073980in,left,base]{\color{textcolor}\rmfamily\fontsize{7.000000}{8.400000}\selectfont u1}%
\end{pgfscope}%
\begin{pgfscope}%
\pgfsetbuttcap%
\pgfsetroundjoin%
\pgfsetlinewidth{1.505625pt}%
\definecolor{currentstroke}{rgb}{0.498039,0.498039,0.498039}%
\pgfsetstrokecolor{currentstroke}%
\pgfsetdash{{5.550000pt}{2.400000pt}}{0.000000pt}%
\pgfpathmoveto{\pgfqpoint{1.363936in}{3.972437in}}%
\pgfpathlineto{\pgfqpoint{1.461158in}{3.972437in}}%
\pgfpathlineto{\pgfqpoint{1.558381in}{3.972437in}}%
\pgfusepath{stroke}%
\end{pgfscope}%
\begin{pgfscope}%
\definecolor{textcolor}{rgb}{0.000000,0.000000,0.000000}%
\pgfsetstrokecolor{textcolor}%
\pgfsetfillcolor{textcolor}%
\pgftext[x=1.636158in,y=3.938409in,left,base]{\color{textcolor}\rmfamily\fontsize{7.000000}{8.400000}\selectfont u4}%
\end{pgfscope}%
\begin{pgfscope}%
\pgfsetbuttcap%
\pgfsetroundjoin%
\pgfsetlinewidth{1.505625pt}%
\definecolor{currentstroke}{rgb}{0.121569,0.466667,0.705882}%
\pgfsetstrokecolor{currentstroke}%
\pgfsetdash{{9.600000pt}{2.400000pt}{1.500000pt}{2.400000pt}}{0.000000pt}%
\pgfpathmoveto{\pgfqpoint{1.363936in}{3.836866in}}%
\pgfpathlineto{\pgfqpoint{1.461158in}{3.836866in}}%
\pgfpathlineto{\pgfqpoint{1.558381in}{3.836866in}}%
\pgfusepath{stroke}%
\end{pgfscope}%
\begin{pgfscope}%
\definecolor{textcolor}{rgb}{0.000000,0.000000,0.000000}%
\pgfsetstrokecolor{textcolor}%
\pgfsetfillcolor{textcolor}%
\pgftext[x=1.636158in,y=3.802838in,left,base]{\color{textcolor}\rmfamily\fontsize{7.000000}{8.400000}\selectfont f}%
\end{pgfscope}%
\begin{pgfscope}%
\pgfsetbuttcap%
\pgfsetroundjoin%
\pgfsetlinewidth{1.505625pt}%
\definecolor{currentstroke}{rgb}{0.890196,0.466667,0.760784}%
\pgfsetstrokecolor{currentstroke}%
\pgfsetdash{{1.500000pt}{2.475000pt}}{0.000000pt}%
\pgfpathmoveto{\pgfqpoint{2.050402in}{4.108008in}}%
\pgfpathlineto{\pgfqpoint{2.147625in}{4.108008in}}%
\pgfpathlineto{\pgfqpoint{2.244847in}{4.108008in}}%
\pgfusepath{stroke}%
\end{pgfscope}%
\begin{pgfscope}%
\definecolor{textcolor}{rgb}{0.000000,0.000000,0.000000}%
\pgfsetstrokecolor{textcolor}%
\pgfsetfillcolor{textcolor}%
\pgftext[x=2.322625in,y=4.073980in,left,base]{\color{textcolor}\rmfamily\fontsize{7.000000}{8.400000}\selectfont d2}%
\end{pgfscope}%
\begin{pgfscope}%
\pgfsetbuttcap%
\pgfsetroundjoin%
\pgfsetlinewidth{1.505625pt}%
\definecolor{currentstroke}{rgb}{0.498039,0.498039,0.498039}%
\pgfsetstrokecolor{currentstroke}%
\pgfsetdash{{9.600000pt}{2.400000pt}{1.500000pt}{2.400000pt}}{0.000000pt}%
\pgfpathmoveto{\pgfqpoint{2.050402in}{3.972437in}}%
\pgfpathlineto{\pgfqpoint{2.147625in}{3.972437in}}%
\pgfpathlineto{\pgfqpoint{2.244847in}{3.972437in}}%
\pgfusepath{stroke}%
\end{pgfscope}%
\begin{pgfscope}%
\definecolor{textcolor}{rgb}{0.000000,0.000000,0.000000}%
\pgfsetstrokecolor{textcolor}%
\pgfsetfillcolor{textcolor}%
\pgftext[x=2.322625in,y=3.938409in,left,base]{\color{textcolor}\rmfamily\fontsize{7.000000}{8.400000}\selectfont u8}%
\end{pgfscope}%
\begin{pgfscope}%
\pgfsetbuttcap%
\pgfsetroundjoin%
\pgfsetlinewidth{1.505625pt}%
\definecolor{currentstroke}{rgb}{0.498039,0.498039,0.498039}%
\pgfsetstrokecolor{currentstroke}%
\pgfsetdash{{1.500000pt}{2.475000pt}}{0.000000pt}%
\pgfpathmoveto{\pgfqpoint{2.050402in}{3.836866in}}%
\pgfpathlineto{\pgfqpoint{2.147625in}{3.836866in}}%
\pgfpathlineto{\pgfqpoint{2.244847in}{3.836866in}}%
\pgfusepath{stroke}%
\end{pgfscope}%
\begin{pgfscope}%
\definecolor{textcolor}{rgb}{0.000000,0.000000,0.000000}%
\pgfsetstrokecolor{textcolor}%
\pgfsetfillcolor{textcolor}%
\pgftext[x=2.322625in,y=3.802838in,left,base]{\color{textcolor}\rmfamily\fontsize{7.000000}{8.400000}\selectfont u2}%
\end{pgfscope}%
\begin{pgfscope}%
\pgfsetbuttcap%
\pgfsetroundjoin%
\pgfsetlinewidth{1.505625pt}%
\definecolor{currentstroke}{rgb}{0.890196,0.466667,0.760784}%
\pgfsetstrokecolor{currentstroke}%
\pgfsetdash{{5.550000pt}{2.400000pt}}{0.000000pt}%
\pgfpathmoveto{\pgfqpoint{2.736869in}{4.108008in}}%
\pgfpathlineto{\pgfqpoint{2.834091in}{4.108008in}}%
\pgfpathlineto{\pgfqpoint{2.931313in}{4.108008in}}%
\pgfusepath{stroke}%
\end{pgfscope}%
\begin{pgfscope}%
\definecolor{textcolor}{rgb}{0.000000,0.000000,0.000000}%
\pgfsetstrokecolor{textcolor}%
\pgfsetfillcolor{textcolor}%
\pgftext[x=3.009091in,y=4.073980in,left,base]{\color{textcolor}\rmfamily\fontsize{7.000000}{8.400000}\selectfont d4}%
\end{pgfscope}%
\begin{pgfscope}%
\pgfsetbuttcap%
\pgfsetroundjoin%
\pgfsetlinewidth{1.505625pt}%
\definecolor{currentstroke}{rgb}{0.890196,0.466667,0.760784}%
\pgfsetstrokecolor{currentstroke}%
\pgfsetdash{{9.600000pt}{2.400000pt}{1.500000pt}{2.400000pt}}{0.000000pt}%
\pgfpathmoveto{\pgfqpoint{2.736869in}{3.972437in}}%
\pgfpathlineto{\pgfqpoint{2.834091in}{3.972437in}}%
\pgfpathlineto{\pgfqpoint{2.931313in}{3.972437in}}%
\pgfusepath{stroke}%
\end{pgfscope}%
\begin{pgfscope}%
\definecolor{textcolor}{rgb}{0.000000,0.000000,0.000000}%
\pgfsetstrokecolor{textcolor}%
\pgfsetfillcolor{textcolor}%
\pgftext[x=3.009091in,y=3.938409in,left,base]{\color{textcolor}\rmfamily\fontsize{7.000000}{8.400000}\selectfont d8}%
\end{pgfscope}%
\begin{pgfscope}%
\pgfsetrectcap%
\pgfsetroundjoin%
\pgfsetlinewidth{1.505625pt}%
\definecolor{currentstroke}{rgb}{0.172549,0.627451,0.172549}%
\pgfsetstrokecolor{currentstroke}%
\pgfsetdash{}{0pt}%
\pgfpathmoveto{\pgfqpoint{2.736869in}{3.836866in}}%
\pgfpathlineto{\pgfqpoint{2.834091in}{3.836866in}}%
\pgfpathlineto{\pgfqpoint{2.931313in}{3.836866in}}%
\pgfusepath{stroke}%
\end{pgfscope}%
\begin{pgfscope}%
\definecolor{textcolor}{rgb}{0.000000,0.000000,0.000000}%
\pgfsetstrokecolor{textcolor}%
\pgfsetfillcolor{textcolor}%
\pgftext[x=3.009091in,y=3.802838in,left,base]{\color{textcolor}\rmfamily\fontsize{7.000000}{8.400000}\selectfont o}%
\end{pgfscope}%
\end{pgfpicture}%
\makeatother%
\endgroup%

  \end{center}
  \caption{Decompression times for files generated with ~od~'s various output formats.}
  \label{fig:od-output}
\end{figure}

We can see that the decompression times vary significantly depending on the
output format chosen. For our nefarious purposes, the best output formats (at
level 6, the default one) are \texttt{a} and \texttt{c} (ASCII named and unnamed), although
\texttt{x} almost matches them.

\subsection{Finding the optimal file}
\label{sec:orgf7c9361}

Finally, the graph in \autoref{fig:all} shows the decompression times for
\texttt{od}'s best output formats along with the other tools.

\begin{figure}[h]
  \begin{center}
    %% Creator: Matplotlib, PGF backend
%%
%% To include the figure in your LaTeX document, write
%%   \input{<filename>.pgf}
%%
%% Make sure the required packages are loaded in your preamble
%%   \usepackage{pgf}
%%
%% Also ensure that all the required font packages are loaded; for instance,
%% the lmodern package is sometimes necessary when using math font.
%%   \usepackage{lmodern}
%%
%% Figures using additional raster images can only be included by \input if
%% they are in the same directory as the main LaTeX file. For loading figures
%% from other directories you can use the `import` package
%%   \usepackage{import}
%%
%% and then include the figures with
%%   \import{<path to file>}{<filename>.pgf}
%%
%% Matplotlib used the following preamble
%%   
%%   \makeatletter\@ifpackageloaded{underscore}{}{\usepackage[strings]{underscore}}\makeatother
%%
\begingroup%
\makeatletter%
\begin{pgfpicture}%
\pgfpathrectangle{\pgfpointorigin}{\pgfqpoint{3.195241in}{5.196975in}}%
\pgfusepath{use as bounding box, clip}%
\begin{pgfscope}%
\pgfsetbuttcap%
\pgfsetmiterjoin%
\definecolor{currentfill}{rgb}{1.000000,1.000000,1.000000}%
\pgfsetfillcolor{currentfill}%
\pgfsetlinewidth{0.000000pt}%
\definecolor{currentstroke}{rgb}{1.000000,1.000000,1.000000}%
\pgfsetstrokecolor{currentstroke}%
\pgfsetdash{}{0pt}%
\pgfpathmoveto{\pgfqpoint{0.000000in}{0.000000in}}%
\pgfpathlineto{\pgfqpoint{3.195241in}{0.000000in}}%
\pgfpathlineto{\pgfqpoint{3.195241in}{5.196975in}}%
\pgfpathlineto{\pgfqpoint{0.000000in}{5.196975in}}%
\pgfpathlineto{\pgfqpoint{0.000000in}{0.000000in}}%
\pgfpathclose%
\pgfusepath{fill}%
\end{pgfscope}%
\begin{pgfscope}%
\pgfsetbuttcap%
\pgfsetmiterjoin%
\definecolor{currentfill}{rgb}{1.000000,1.000000,1.000000}%
\pgfsetfillcolor{currentfill}%
\pgfsetlinewidth{0.000000pt}%
\definecolor{currentstroke}{rgb}{0.000000,0.000000,0.000000}%
\pgfsetstrokecolor{currentstroke}%
\pgfsetstrokeopacity{0.000000}%
\pgfsetdash{}{0pt}%
\pgfpathmoveto{\pgfqpoint{0.569136in}{0.499691in}}%
\pgfpathlineto{\pgfqpoint{3.095241in}{0.499691in}}%
\pgfpathlineto{\pgfqpoint{3.095241in}{3.548498in}}%
\pgfpathlineto{\pgfqpoint{0.569136in}{3.548498in}}%
\pgfpathlineto{\pgfqpoint{0.569136in}{0.499691in}}%
\pgfpathclose%
\pgfusepath{fill}%
\end{pgfscope}%
\begin{pgfscope}%
\pgfsetbuttcap%
\pgfsetroundjoin%
\definecolor{currentfill}{rgb}{0.000000,0.000000,0.000000}%
\pgfsetfillcolor{currentfill}%
\pgfsetlinewidth{0.803000pt}%
\definecolor{currentstroke}{rgb}{0.000000,0.000000,0.000000}%
\pgfsetstrokecolor{currentstroke}%
\pgfsetdash{}{0pt}%
\pgfsys@defobject{currentmarker}{\pgfqpoint{0.000000in}{-0.048611in}}{\pgfqpoint{0.000000in}{0.000000in}}{%
\pgfpathmoveto{\pgfqpoint{0.000000in}{0.000000in}}%
\pgfpathlineto{\pgfqpoint{0.000000in}{-0.048611in}}%
\pgfusepath{stroke,fill}%
}%
\begin{pgfscope}%
\pgfsys@transformshift{0.683959in}{0.499691in}%
\pgfsys@useobject{currentmarker}{}%
\end{pgfscope}%
\end{pgfscope}%
\begin{pgfscope}%
\definecolor{textcolor}{rgb}{0.000000,0.000000,0.000000}%
\pgfsetstrokecolor{textcolor}%
\pgfsetfillcolor{textcolor}%
\pgftext[x=0.683959in,y=0.402469in,,top]{\color{textcolor}\rmfamily\fontsize{10.000000}{12.000000}\selectfont \(\displaystyle {1}\)}%
\end{pgfscope}%
\begin{pgfscope}%
\pgfsetbuttcap%
\pgfsetroundjoin%
\definecolor{currentfill}{rgb}{0.000000,0.000000,0.000000}%
\pgfsetfillcolor{currentfill}%
\pgfsetlinewidth{0.803000pt}%
\definecolor{currentstroke}{rgb}{0.000000,0.000000,0.000000}%
\pgfsetstrokecolor{currentstroke}%
\pgfsetdash{}{0pt}%
\pgfsys@defobject{currentmarker}{\pgfqpoint{0.000000in}{-0.048611in}}{\pgfqpoint{0.000000in}{0.000000in}}{%
\pgfpathmoveto{\pgfqpoint{0.000000in}{0.000000in}}%
\pgfpathlineto{\pgfqpoint{0.000000in}{-0.048611in}}%
\pgfusepath{stroke,fill}%
}%
\begin{pgfscope}%
\pgfsys@transformshift{0.971017in}{0.499691in}%
\pgfsys@useobject{currentmarker}{}%
\end{pgfscope}%
\end{pgfscope}%
\begin{pgfscope}%
\definecolor{textcolor}{rgb}{0.000000,0.000000,0.000000}%
\pgfsetstrokecolor{textcolor}%
\pgfsetfillcolor{textcolor}%
\pgftext[x=0.971017in,y=0.402469in,,top]{\color{textcolor}\rmfamily\fontsize{10.000000}{12.000000}\selectfont \(\displaystyle {2}\)}%
\end{pgfscope}%
\begin{pgfscope}%
\pgfsetbuttcap%
\pgfsetroundjoin%
\definecolor{currentfill}{rgb}{0.000000,0.000000,0.000000}%
\pgfsetfillcolor{currentfill}%
\pgfsetlinewidth{0.803000pt}%
\definecolor{currentstroke}{rgb}{0.000000,0.000000,0.000000}%
\pgfsetstrokecolor{currentstroke}%
\pgfsetdash{}{0pt}%
\pgfsys@defobject{currentmarker}{\pgfqpoint{0.000000in}{-0.048611in}}{\pgfqpoint{0.000000in}{0.000000in}}{%
\pgfpathmoveto{\pgfqpoint{0.000000in}{0.000000in}}%
\pgfpathlineto{\pgfqpoint{0.000000in}{-0.048611in}}%
\pgfusepath{stroke,fill}%
}%
\begin{pgfscope}%
\pgfsys@transformshift{1.258074in}{0.499691in}%
\pgfsys@useobject{currentmarker}{}%
\end{pgfscope}%
\end{pgfscope}%
\begin{pgfscope}%
\definecolor{textcolor}{rgb}{0.000000,0.000000,0.000000}%
\pgfsetstrokecolor{textcolor}%
\pgfsetfillcolor{textcolor}%
\pgftext[x=1.258074in,y=0.402469in,,top]{\color{textcolor}\rmfamily\fontsize{10.000000}{12.000000}\selectfont \(\displaystyle {3}\)}%
\end{pgfscope}%
\begin{pgfscope}%
\pgfsetbuttcap%
\pgfsetroundjoin%
\definecolor{currentfill}{rgb}{0.000000,0.000000,0.000000}%
\pgfsetfillcolor{currentfill}%
\pgfsetlinewidth{0.803000pt}%
\definecolor{currentstroke}{rgb}{0.000000,0.000000,0.000000}%
\pgfsetstrokecolor{currentstroke}%
\pgfsetdash{}{0pt}%
\pgfsys@defobject{currentmarker}{\pgfqpoint{0.000000in}{-0.048611in}}{\pgfqpoint{0.000000in}{0.000000in}}{%
\pgfpathmoveto{\pgfqpoint{0.000000in}{0.000000in}}%
\pgfpathlineto{\pgfqpoint{0.000000in}{-0.048611in}}%
\pgfusepath{stroke,fill}%
}%
\begin{pgfscope}%
\pgfsys@transformshift{1.545131in}{0.499691in}%
\pgfsys@useobject{currentmarker}{}%
\end{pgfscope}%
\end{pgfscope}%
\begin{pgfscope}%
\definecolor{textcolor}{rgb}{0.000000,0.000000,0.000000}%
\pgfsetstrokecolor{textcolor}%
\pgfsetfillcolor{textcolor}%
\pgftext[x=1.545131in,y=0.402469in,,top]{\color{textcolor}\rmfamily\fontsize{10.000000}{12.000000}\selectfont \(\displaystyle {4}\)}%
\end{pgfscope}%
\begin{pgfscope}%
\pgfsetbuttcap%
\pgfsetroundjoin%
\definecolor{currentfill}{rgb}{0.000000,0.000000,0.000000}%
\pgfsetfillcolor{currentfill}%
\pgfsetlinewidth{0.803000pt}%
\definecolor{currentstroke}{rgb}{0.000000,0.000000,0.000000}%
\pgfsetstrokecolor{currentstroke}%
\pgfsetdash{}{0pt}%
\pgfsys@defobject{currentmarker}{\pgfqpoint{0.000000in}{-0.048611in}}{\pgfqpoint{0.000000in}{0.000000in}}{%
\pgfpathmoveto{\pgfqpoint{0.000000in}{0.000000in}}%
\pgfpathlineto{\pgfqpoint{0.000000in}{-0.048611in}}%
\pgfusepath{stroke,fill}%
}%
\begin{pgfscope}%
\pgfsys@transformshift{1.832189in}{0.499691in}%
\pgfsys@useobject{currentmarker}{}%
\end{pgfscope}%
\end{pgfscope}%
\begin{pgfscope}%
\definecolor{textcolor}{rgb}{0.000000,0.000000,0.000000}%
\pgfsetstrokecolor{textcolor}%
\pgfsetfillcolor{textcolor}%
\pgftext[x=1.832189in,y=0.402469in,,top]{\color{textcolor}\rmfamily\fontsize{10.000000}{12.000000}\selectfont \(\displaystyle {5}\)}%
\end{pgfscope}%
\begin{pgfscope}%
\pgfsetbuttcap%
\pgfsetroundjoin%
\definecolor{currentfill}{rgb}{0.000000,0.000000,0.000000}%
\pgfsetfillcolor{currentfill}%
\pgfsetlinewidth{0.803000pt}%
\definecolor{currentstroke}{rgb}{0.000000,0.000000,0.000000}%
\pgfsetstrokecolor{currentstroke}%
\pgfsetdash{}{0pt}%
\pgfsys@defobject{currentmarker}{\pgfqpoint{0.000000in}{-0.048611in}}{\pgfqpoint{0.000000in}{0.000000in}}{%
\pgfpathmoveto{\pgfqpoint{0.000000in}{0.000000in}}%
\pgfpathlineto{\pgfqpoint{0.000000in}{-0.048611in}}%
\pgfusepath{stroke,fill}%
}%
\begin{pgfscope}%
\pgfsys@transformshift{2.119246in}{0.499691in}%
\pgfsys@useobject{currentmarker}{}%
\end{pgfscope}%
\end{pgfscope}%
\begin{pgfscope}%
\definecolor{textcolor}{rgb}{0.000000,0.000000,0.000000}%
\pgfsetstrokecolor{textcolor}%
\pgfsetfillcolor{textcolor}%
\pgftext[x=2.119246in,y=0.402469in,,top]{\color{textcolor}\rmfamily\fontsize{10.000000}{12.000000}\selectfont \(\displaystyle {6}\)}%
\end{pgfscope}%
\begin{pgfscope}%
\pgfsetbuttcap%
\pgfsetroundjoin%
\definecolor{currentfill}{rgb}{0.000000,0.000000,0.000000}%
\pgfsetfillcolor{currentfill}%
\pgfsetlinewidth{0.803000pt}%
\definecolor{currentstroke}{rgb}{0.000000,0.000000,0.000000}%
\pgfsetstrokecolor{currentstroke}%
\pgfsetdash{}{0pt}%
\pgfsys@defobject{currentmarker}{\pgfqpoint{0.000000in}{-0.048611in}}{\pgfqpoint{0.000000in}{0.000000in}}{%
\pgfpathmoveto{\pgfqpoint{0.000000in}{0.000000in}}%
\pgfpathlineto{\pgfqpoint{0.000000in}{-0.048611in}}%
\pgfusepath{stroke,fill}%
}%
\begin{pgfscope}%
\pgfsys@transformshift{2.406303in}{0.499691in}%
\pgfsys@useobject{currentmarker}{}%
\end{pgfscope}%
\end{pgfscope}%
\begin{pgfscope}%
\definecolor{textcolor}{rgb}{0.000000,0.000000,0.000000}%
\pgfsetstrokecolor{textcolor}%
\pgfsetfillcolor{textcolor}%
\pgftext[x=2.406303in,y=0.402469in,,top]{\color{textcolor}\rmfamily\fontsize{10.000000}{12.000000}\selectfont \(\displaystyle {7}\)}%
\end{pgfscope}%
\begin{pgfscope}%
\pgfsetbuttcap%
\pgfsetroundjoin%
\definecolor{currentfill}{rgb}{0.000000,0.000000,0.000000}%
\pgfsetfillcolor{currentfill}%
\pgfsetlinewidth{0.803000pt}%
\definecolor{currentstroke}{rgb}{0.000000,0.000000,0.000000}%
\pgfsetstrokecolor{currentstroke}%
\pgfsetdash{}{0pt}%
\pgfsys@defobject{currentmarker}{\pgfqpoint{0.000000in}{-0.048611in}}{\pgfqpoint{0.000000in}{0.000000in}}{%
\pgfpathmoveto{\pgfqpoint{0.000000in}{0.000000in}}%
\pgfpathlineto{\pgfqpoint{0.000000in}{-0.048611in}}%
\pgfusepath{stroke,fill}%
}%
\begin{pgfscope}%
\pgfsys@transformshift{2.693361in}{0.499691in}%
\pgfsys@useobject{currentmarker}{}%
\end{pgfscope}%
\end{pgfscope}%
\begin{pgfscope}%
\definecolor{textcolor}{rgb}{0.000000,0.000000,0.000000}%
\pgfsetstrokecolor{textcolor}%
\pgfsetfillcolor{textcolor}%
\pgftext[x=2.693361in,y=0.402469in,,top]{\color{textcolor}\rmfamily\fontsize{10.000000}{12.000000}\selectfont \(\displaystyle {8}\)}%
\end{pgfscope}%
\begin{pgfscope}%
\pgfsetbuttcap%
\pgfsetroundjoin%
\definecolor{currentfill}{rgb}{0.000000,0.000000,0.000000}%
\pgfsetfillcolor{currentfill}%
\pgfsetlinewidth{0.803000pt}%
\definecolor{currentstroke}{rgb}{0.000000,0.000000,0.000000}%
\pgfsetstrokecolor{currentstroke}%
\pgfsetdash{}{0pt}%
\pgfsys@defobject{currentmarker}{\pgfqpoint{0.000000in}{-0.048611in}}{\pgfqpoint{0.000000in}{0.000000in}}{%
\pgfpathmoveto{\pgfqpoint{0.000000in}{0.000000in}}%
\pgfpathlineto{\pgfqpoint{0.000000in}{-0.048611in}}%
\pgfusepath{stroke,fill}%
}%
\begin{pgfscope}%
\pgfsys@transformshift{2.980418in}{0.499691in}%
\pgfsys@useobject{currentmarker}{}%
\end{pgfscope}%
\end{pgfscope}%
\begin{pgfscope}%
\definecolor{textcolor}{rgb}{0.000000,0.000000,0.000000}%
\pgfsetstrokecolor{textcolor}%
\pgfsetfillcolor{textcolor}%
\pgftext[x=2.980418in,y=0.402469in,,top]{\color{textcolor}\rmfamily\fontsize{10.000000}{12.000000}\selectfont \(\displaystyle {9}\)}%
\end{pgfscope}%
\begin{pgfscope}%
\definecolor{textcolor}{rgb}{0.000000,0.000000,0.000000}%
\pgfsetstrokecolor{textcolor}%
\pgfsetfillcolor{textcolor}%
\pgftext[x=1.832189in,y=0.223457in,,top]{\color{textcolor}\rmfamily\fontsize{10.000000}{12.000000}\selectfont gzip compression level}%
\end{pgfscope}%
\begin{pgfscope}%
\pgfsetbuttcap%
\pgfsetroundjoin%
\definecolor{currentfill}{rgb}{0.000000,0.000000,0.000000}%
\pgfsetfillcolor{currentfill}%
\pgfsetlinewidth{0.803000pt}%
\definecolor{currentstroke}{rgb}{0.000000,0.000000,0.000000}%
\pgfsetstrokecolor{currentstroke}%
\pgfsetdash{}{0pt}%
\pgfsys@defobject{currentmarker}{\pgfqpoint{-0.048611in}{0.000000in}}{\pgfqpoint{-0.000000in}{0.000000in}}{%
\pgfpathmoveto{\pgfqpoint{-0.000000in}{0.000000in}}%
\pgfpathlineto{\pgfqpoint{-0.048611in}{0.000000in}}%
\pgfusepath{stroke,fill}%
}%
\begin{pgfscope}%
\pgfsys@transformshift{0.569136in}{0.711520in}%
\pgfsys@useobject{currentmarker}{}%
\end{pgfscope}%
\end{pgfscope}%
\begin{pgfscope}%
\definecolor{textcolor}{rgb}{0.000000,0.000000,0.000000}%
\pgfsetstrokecolor{textcolor}%
\pgfsetfillcolor{textcolor}%
\pgftext[x=0.294444in, y=0.663294in, left, base]{\color{textcolor}\rmfamily\fontsize{10.000000}{12.000000}\selectfont \(\displaystyle {0.3}\)}%
\end{pgfscope}%
\begin{pgfscope}%
\pgfsetbuttcap%
\pgfsetroundjoin%
\definecolor{currentfill}{rgb}{0.000000,0.000000,0.000000}%
\pgfsetfillcolor{currentfill}%
\pgfsetlinewidth{0.803000pt}%
\definecolor{currentstroke}{rgb}{0.000000,0.000000,0.000000}%
\pgfsetstrokecolor{currentstroke}%
\pgfsetdash{}{0pt}%
\pgfsys@defobject{currentmarker}{\pgfqpoint{-0.048611in}{0.000000in}}{\pgfqpoint{-0.000000in}{0.000000in}}{%
\pgfpathmoveto{\pgfqpoint{-0.000000in}{0.000000in}}%
\pgfpathlineto{\pgfqpoint{-0.048611in}{0.000000in}}%
\pgfusepath{stroke,fill}%
}%
\begin{pgfscope}%
\pgfsys@transformshift{0.569136in}{1.150998in}%
\pgfsys@useobject{currentmarker}{}%
\end{pgfscope}%
\end{pgfscope}%
\begin{pgfscope}%
\definecolor{textcolor}{rgb}{0.000000,0.000000,0.000000}%
\pgfsetstrokecolor{textcolor}%
\pgfsetfillcolor{textcolor}%
\pgftext[x=0.294444in, y=1.102773in, left, base]{\color{textcolor}\rmfamily\fontsize{10.000000}{12.000000}\selectfont \(\displaystyle {0.4}\)}%
\end{pgfscope}%
\begin{pgfscope}%
\pgfsetbuttcap%
\pgfsetroundjoin%
\definecolor{currentfill}{rgb}{0.000000,0.000000,0.000000}%
\pgfsetfillcolor{currentfill}%
\pgfsetlinewidth{0.803000pt}%
\definecolor{currentstroke}{rgb}{0.000000,0.000000,0.000000}%
\pgfsetstrokecolor{currentstroke}%
\pgfsetdash{}{0pt}%
\pgfsys@defobject{currentmarker}{\pgfqpoint{-0.048611in}{0.000000in}}{\pgfqpoint{-0.000000in}{0.000000in}}{%
\pgfpathmoveto{\pgfqpoint{-0.000000in}{0.000000in}}%
\pgfpathlineto{\pgfqpoint{-0.048611in}{0.000000in}}%
\pgfusepath{stroke,fill}%
}%
\begin{pgfscope}%
\pgfsys@transformshift{0.569136in}{1.590476in}%
\pgfsys@useobject{currentmarker}{}%
\end{pgfscope}%
\end{pgfscope}%
\begin{pgfscope}%
\definecolor{textcolor}{rgb}{0.000000,0.000000,0.000000}%
\pgfsetstrokecolor{textcolor}%
\pgfsetfillcolor{textcolor}%
\pgftext[x=0.294444in, y=1.542251in, left, base]{\color{textcolor}\rmfamily\fontsize{10.000000}{12.000000}\selectfont \(\displaystyle {0.5}\)}%
\end{pgfscope}%
\begin{pgfscope}%
\pgfsetbuttcap%
\pgfsetroundjoin%
\definecolor{currentfill}{rgb}{0.000000,0.000000,0.000000}%
\pgfsetfillcolor{currentfill}%
\pgfsetlinewidth{0.803000pt}%
\definecolor{currentstroke}{rgb}{0.000000,0.000000,0.000000}%
\pgfsetstrokecolor{currentstroke}%
\pgfsetdash{}{0pt}%
\pgfsys@defobject{currentmarker}{\pgfqpoint{-0.048611in}{0.000000in}}{\pgfqpoint{-0.000000in}{0.000000in}}{%
\pgfpathmoveto{\pgfqpoint{-0.000000in}{0.000000in}}%
\pgfpathlineto{\pgfqpoint{-0.048611in}{0.000000in}}%
\pgfusepath{stroke,fill}%
}%
\begin{pgfscope}%
\pgfsys@transformshift{0.569136in}{2.029954in}%
\pgfsys@useobject{currentmarker}{}%
\end{pgfscope}%
\end{pgfscope}%
\begin{pgfscope}%
\definecolor{textcolor}{rgb}{0.000000,0.000000,0.000000}%
\pgfsetstrokecolor{textcolor}%
\pgfsetfillcolor{textcolor}%
\pgftext[x=0.294444in, y=1.981729in, left, base]{\color{textcolor}\rmfamily\fontsize{10.000000}{12.000000}\selectfont \(\displaystyle {0.6}\)}%
\end{pgfscope}%
\begin{pgfscope}%
\pgfsetbuttcap%
\pgfsetroundjoin%
\definecolor{currentfill}{rgb}{0.000000,0.000000,0.000000}%
\pgfsetfillcolor{currentfill}%
\pgfsetlinewidth{0.803000pt}%
\definecolor{currentstroke}{rgb}{0.000000,0.000000,0.000000}%
\pgfsetstrokecolor{currentstroke}%
\pgfsetdash{}{0pt}%
\pgfsys@defobject{currentmarker}{\pgfqpoint{-0.048611in}{0.000000in}}{\pgfqpoint{-0.000000in}{0.000000in}}{%
\pgfpathmoveto{\pgfqpoint{-0.000000in}{0.000000in}}%
\pgfpathlineto{\pgfqpoint{-0.048611in}{0.000000in}}%
\pgfusepath{stroke,fill}%
}%
\begin{pgfscope}%
\pgfsys@transformshift{0.569136in}{2.469433in}%
\pgfsys@useobject{currentmarker}{}%
\end{pgfscope}%
\end{pgfscope}%
\begin{pgfscope}%
\definecolor{textcolor}{rgb}{0.000000,0.000000,0.000000}%
\pgfsetstrokecolor{textcolor}%
\pgfsetfillcolor{textcolor}%
\pgftext[x=0.294444in, y=2.421207in, left, base]{\color{textcolor}\rmfamily\fontsize{10.000000}{12.000000}\selectfont \(\displaystyle {0.7}\)}%
\end{pgfscope}%
\begin{pgfscope}%
\pgfsetbuttcap%
\pgfsetroundjoin%
\definecolor{currentfill}{rgb}{0.000000,0.000000,0.000000}%
\pgfsetfillcolor{currentfill}%
\pgfsetlinewidth{0.803000pt}%
\definecolor{currentstroke}{rgb}{0.000000,0.000000,0.000000}%
\pgfsetstrokecolor{currentstroke}%
\pgfsetdash{}{0pt}%
\pgfsys@defobject{currentmarker}{\pgfqpoint{-0.048611in}{0.000000in}}{\pgfqpoint{-0.000000in}{0.000000in}}{%
\pgfpathmoveto{\pgfqpoint{-0.000000in}{0.000000in}}%
\pgfpathlineto{\pgfqpoint{-0.048611in}{0.000000in}}%
\pgfusepath{stroke,fill}%
}%
\begin{pgfscope}%
\pgfsys@transformshift{0.569136in}{2.908911in}%
\pgfsys@useobject{currentmarker}{}%
\end{pgfscope}%
\end{pgfscope}%
\begin{pgfscope}%
\definecolor{textcolor}{rgb}{0.000000,0.000000,0.000000}%
\pgfsetstrokecolor{textcolor}%
\pgfsetfillcolor{textcolor}%
\pgftext[x=0.294444in, y=2.860686in, left, base]{\color{textcolor}\rmfamily\fontsize{10.000000}{12.000000}\selectfont \(\displaystyle {0.8}\)}%
\end{pgfscope}%
\begin{pgfscope}%
\pgfsetbuttcap%
\pgfsetroundjoin%
\definecolor{currentfill}{rgb}{0.000000,0.000000,0.000000}%
\pgfsetfillcolor{currentfill}%
\pgfsetlinewidth{0.803000pt}%
\definecolor{currentstroke}{rgb}{0.000000,0.000000,0.000000}%
\pgfsetstrokecolor{currentstroke}%
\pgfsetdash{}{0pt}%
\pgfsys@defobject{currentmarker}{\pgfqpoint{-0.048611in}{0.000000in}}{\pgfqpoint{-0.000000in}{0.000000in}}{%
\pgfpathmoveto{\pgfqpoint{-0.000000in}{0.000000in}}%
\pgfpathlineto{\pgfqpoint{-0.048611in}{0.000000in}}%
\pgfusepath{stroke,fill}%
}%
\begin{pgfscope}%
\pgfsys@transformshift{0.569136in}{3.348389in}%
\pgfsys@useobject{currentmarker}{}%
\end{pgfscope}%
\end{pgfscope}%
\begin{pgfscope}%
\definecolor{textcolor}{rgb}{0.000000,0.000000,0.000000}%
\pgfsetstrokecolor{textcolor}%
\pgfsetfillcolor{textcolor}%
\pgftext[x=0.294444in, y=3.300164in, left, base]{\color{textcolor}\rmfamily\fontsize{10.000000}{12.000000}\selectfont \(\displaystyle {0.9}\)}%
\end{pgfscope}%
\begin{pgfscope}%
\definecolor{textcolor}{rgb}{0.000000,0.000000,0.000000}%
\pgfsetstrokecolor{textcolor}%
\pgfsetfillcolor{textcolor}%
\pgftext[x=0.238889in,y=2.024095in,,bottom,rotate=90.000000]{\color{textcolor}\rmfamily\fontsize{10.000000}{12.000000}\selectfont decompression time (s)}%
\end{pgfscope}%
\begin{pgfscope}%
\pgfpathrectangle{\pgfqpoint{0.569136in}{0.499691in}}{\pgfqpoint{2.526105in}{3.048807in}}%
\pgfusepath{clip}%
\pgfsetrectcap%
\pgfsetroundjoin%
\pgfsetlinewidth{1.505625pt}%
\definecolor{currentstroke}{rgb}{0.090196,0.745098,0.811765}%
\pgfsetstrokecolor{currentstroke}%
\pgfsetdash{}{0pt}%
\pgfpathmoveto{\pgfqpoint{0.683959in}{3.374758in}}%
\pgfpathlineto{\pgfqpoint{0.971017in}{3.409916in}}%
\pgfpathlineto{\pgfqpoint{1.258074in}{3.181388in}}%
\pgfpathlineto{\pgfqpoint{1.545131in}{3.172598in}}%
\pgfpathlineto{\pgfqpoint{1.832189in}{3.119861in}}%
\pgfpathlineto{\pgfqpoint{2.119246in}{3.023175in}}%
\pgfpathlineto{\pgfqpoint{2.406303in}{3.031965in}}%
\pgfpathlineto{\pgfqpoint{2.693361in}{2.996807in}}%
\pgfpathlineto{\pgfqpoint{2.980418in}{3.040755in}}%
\pgfusepath{stroke}%
\end{pgfscope}%
\begin{pgfscope}%
\pgfpathrectangle{\pgfqpoint{0.569136in}{0.499691in}}{\pgfqpoint{2.526105in}{3.048807in}}%
\pgfusepath{clip}%
\pgfsetrectcap%
\pgfsetroundjoin%
\pgfsetlinewidth{1.505625pt}%
\definecolor{currentstroke}{rgb}{0.839216,0.152941,0.156863}%
\pgfsetstrokecolor{currentstroke}%
\pgfsetdash{}{0pt}%
\pgfpathmoveto{\pgfqpoint{0.683959in}{2.706751in}}%
\pgfpathlineto{\pgfqpoint{0.971017in}{2.821015in}}%
\pgfpathlineto{\pgfqpoint{1.258074in}{2.785857in}}%
\pgfpathlineto{\pgfqpoint{1.545131in}{3.023175in}}%
\pgfpathlineto{\pgfqpoint{1.832189in}{3.040755in}}%
\pgfpathlineto{\pgfqpoint{2.119246in}{2.970438in}}%
\pgfpathlineto{\pgfqpoint{2.406303in}{2.952859in}}%
\pgfpathlineto{\pgfqpoint{2.693361in}{2.979228in}}%
\pgfpathlineto{\pgfqpoint{2.980418in}{3.163808in}}%
\pgfusepath{stroke}%
\end{pgfscope}%
\begin{pgfscope}%
\pgfpathrectangle{\pgfqpoint{0.569136in}{0.499691in}}{\pgfqpoint{2.526105in}{3.048807in}}%
\pgfusepath{clip}%
\pgfsetrectcap%
\pgfsetroundjoin%
\pgfsetlinewidth{1.505625pt}%
\definecolor{currentstroke}{rgb}{0.737255,0.741176,0.133333}%
\pgfsetstrokecolor{currentstroke}%
\pgfsetdash{}{0pt}%
\pgfpathmoveto{\pgfqpoint{0.683959in}{3.295652in}}%
\pgfpathlineto{\pgfqpoint{0.971017in}{3.357179in}}%
\pgfpathlineto{\pgfqpoint{1.258074in}{3.392337in}}%
\pgfpathlineto{\pgfqpoint{1.545131in}{3.014386in}}%
\pgfpathlineto{\pgfqpoint{1.832189in}{2.996807in}}%
\pgfpathlineto{\pgfqpoint{2.119246in}{2.961648in}}%
\pgfpathlineto{\pgfqpoint{2.406303in}{2.988017in}}%
\pgfpathlineto{\pgfqpoint{2.693361in}{2.944069in}}%
\pgfpathlineto{\pgfqpoint{2.980418in}{2.908911in}}%
\pgfusepath{stroke}%
\end{pgfscope}%
\begin{pgfscope}%
\pgfpathrectangle{\pgfqpoint{0.569136in}{0.499691in}}{\pgfqpoint{2.526105in}{3.048807in}}%
\pgfusepath{clip}%
\pgfsetrectcap%
\pgfsetroundjoin%
\pgfsetlinewidth{1.505625pt}%
\definecolor{currentstroke}{rgb}{0.890196,0.466667,0.760784}%
\pgfsetstrokecolor{currentstroke}%
\pgfsetdash{}{0pt}%
\pgfpathmoveto{\pgfqpoint{0.683959in}{3.058334in}}%
\pgfpathlineto{\pgfqpoint{0.971017in}{3.005596in}}%
\pgfpathlineto{\pgfqpoint{1.258074in}{2.662803in}}%
\pgfpathlineto{\pgfqpoint{1.545131in}{2.697961in}}%
\pgfpathlineto{\pgfqpoint{1.832189in}{2.724330in}}%
\pgfpathlineto{\pgfqpoint{2.119246in}{2.697961in}}%
\pgfpathlineto{\pgfqpoint{2.406303in}{2.689172in}}%
\pgfpathlineto{\pgfqpoint{2.693361in}{2.794647in}}%
\pgfpathlineto{\pgfqpoint{2.980418in}{2.768278in}}%
\pgfusepath{stroke}%
\end{pgfscope}%
\begin{pgfscope}%
\pgfpathrectangle{\pgfqpoint{0.569136in}{0.499691in}}{\pgfqpoint{2.526105in}{3.048807in}}%
\pgfusepath{clip}%
\pgfsetbuttcap%
\pgfsetroundjoin%
\pgfsetlinewidth{1.505625pt}%
\definecolor{currentstroke}{rgb}{0.121569,0.466667,0.705882}%
\pgfsetstrokecolor{currentstroke}%
\pgfsetdash{{9.600000pt}{2.400000pt}{1.500000pt}{2.400000pt}}{0.000000pt}%
\pgfpathmoveto{\pgfqpoint{0.683959in}{3.031965in}}%
\pgfpathlineto{\pgfqpoint{0.971017in}{3.005596in}}%
\pgfpathlineto{\pgfqpoint{1.258074in}{2.530960in}}%
\pgfpathlineto{\pgfqpoint{1.545131in}{2.548539in}}%
\pgfpathlineto{\pgfqpoint{1.832189in}{2.636434in}}%
\pgfpathlineto{\pgfqpoint{2.119246in}{2.601276in}}%
\pgfpathlineto{\pgfqpoint{2.406303in}{2.724330in}}%
\pgfpathlineto{\pgfqpoint{2.693361in}{2.697961in}}%
\pgfpathlineto{\pgfqpoint{2.980418in}{2.627645in}}%
\pgfusepath{stroke}%
\end{pgfscope}%
\begin{pgfscope}%
\pgfpathrectangle{\pgfqpoint{0.569136in}{0.499691in}}{\pgfqpoint{2.526105in}{3.048807in}}%
\pgfusepath{clip}%
\pgfsetrectcap%
\pgfsetroundjoin%
\pgfsetlinewidth{1.505625pt}%
\definecolor{currentstroke}{rgb}{0.172549,0.627451,0.172549}%
\pgfsetstrokecolor{currentstroke}%
\pgfsetdash{}{0pt}%
\pgfpathmoveto{\pgfqpoint{0.683959in}{2.864963in}}%
\pgfpathlineto{\pgfqpoint{0.971017in}{2.900121in}}%
\pgfpathlineto{\pgfqpoint{1.258074in}{2.856174in}}%
\pgfpathlineto{\pgfqpoint{1.545131in}{2.460643in}}%
\pgfpathlineto{\pgfqpoint{1.832189in}{2.451854in}}%
\pgfpathlineto{\pgfqpoint{2.119246in}{2.416695in}}%
\pgfpathlineto{\pgfqpoint{2.406303in}{2.425485in}}%
\pgfpathlineto{\pgfqpoint{2.693361in}{2.416695in}}%
\pgfpathlineto{\pgfqpoint{2.980418in}{2.434274in}}%
\pgfusepath{stroke}%
\end{pgfscope}%
\begin{pgfscope}%
\pgfpathrectangle{\pgfqpoint{0.569136in}{0.499691in}}{\pgfqpoint{2.526105in}{3.048807in}}%
\pgfusepath{clip}%
\pgfsetrectcap%
\pgfsetroundjoin%
\pgfsetlinewidth{1.505625pt}%
\definecolor{currentstroke}{rgb}{0.580392,0.403922,0.741176}%
\pgfsetstrokecolor{currentstroke}%
\pgfsetdash{}{0pt}%
\pgfpathmoveto{\pgfqpoint{0.683959in}{2.803436in}}%
\pgfpathlineto{\pgfqpoint{0.971017in}{2.803436in}}%
\pgfpathlineto{\pgfqpoint{1.258074in}{2.812226in}}%
\pgfpathlineto{\pgfqpoint{1.545131in}{2.337589in}}%
\pgfpathlineto{\pgfqpoint{1.832189in}{2.346379in}}%
\pgfpathlineto{\pgfqpoint{2.119246in}{2.337589in}}%
\pgfpathlineto{\pgfqpoint{2.406303in}{2.337589in}}%
\pgfpathlineto{\pgfqpoint{2.693361in}{2.363958in}}%
\pgfpathlineto{\pgfqpoint{2.980418in}{2.346379in}}%
\pgfusepath{stroke}%
\end{pgfscope}%
\begin{pgfscope}%
\pgfpathrectangle{\pgfqpoint{0.569136in}{0.499691in}}{\pgfqpoint{2.526105in}{3.048807in}}%
\pgfusepath{clip}%
\pgfsetrectcap%
\pgfsetroundjoin%
\pgfsetlinewidth{1.505625pt}%
\definecolor{currentstroke}{rgb}{1.000000,0.498039,0.054902}%
\pgfsetstrokecolor{currentstroke}%
\pgfsetdash{}{0pt}%
\pgfpathmoveto{\pgfqpoint{0.683959in}{1.282841in}}%
\pgfpathlineto{\pgfqpoint{0.971017in}{1.291631in}}%
\pgfpathlineto{\pgfqpoint{1.258074in}{1.282841in}}%
\pgfpathlineto{\pgfqpoint{1.545131in}{1.370737in}}%
\pgfpathlineto{\pgfqpoint{1.832189in}{1.291631in}}%
\pgfpathlineto{\pgfqpoint{2.119246in}{1.309210in}}%
\pgfpathlineto{\pgfqpoint{2.406303in}{1.344368in}}%
\pgfpathlineto{\pgfqpoint{2.693361in}{1.309210in}}%
\pgfpathlineto{\pgfqpoint{2.980418in}{1.309210in}}%
\pgfusepath{stroke}%
\end{pgfscope}%
\begin{pgfscope}%
\pgfpathrectangle{\pgfqpoint{0.569136in}{0.499691in}}{\pgfqpoint{2.526105in}{3.048807in}}%
\pgfusepath{clip}%
\pgfsetrectcap%
\pgfsetroundjoin%
\pgfsetlinewidth{1.505625pt}%
\definecolor{currentstroke}{rgb}{0.549020,0.337255,0.294118}%
\pgfsetstrokecolor{currentstroke}%
\pgfsetdash{}{0pt}%
\pgfpathmoveto{\pgfqpoint{0.683959in}{1.238894in}}%
\pgfpathlineto{\pgfqpoint{0.971017in}{1.238894in}}%
\pgfpathlineto{\pgfqpoint{1.258074in}{1.256473in}}%
\pgfpathlineto{\pgfqpoint{1.545131in}{1.247683in}}%
\pgfpathlineto{\pgfqpoint{1.832189in}{1.256473in}}%
\pgfpathlineto{\pgfqpoint{2.119246in}{1.265262in}}%
\pgfpathlineto{\pgfqpoint{2.406303in}{1.300421in}}%
\pgfpathlineto{\pgfqpoint{2.693361in}{1.282841in}}%
\pgfpathlineto{\pgfqpoint{2.980418in}{1.309210in}}%
\pgfusepath{stroke}%
\end{pgfscope}%
\begin{pgfscope}%
\pgfpathrectangle{\pgfqpoint{0.569136in}{0.499691in}}{\pgfqpoint{2.526105in}{3.048807in}}%
\pgfusepath{clip}%
\pgfsetrectcap%
\pgfsetroundjoin%
\pgfsetlinewidth{1.505625pt}%
\definecolor{currentstroke}{rgb}{0.121569,0.466667,0.705882}%
\pgfsetstrokecolor{currentstroke}%
\pgfsetdash{}{0pt}%
\pgfpathmoveto{\pgfqpoint{0.683959in}{0.674896in}}%
\pgfpathlineto{\pgfqpoint{0.971017in}{0.682221in}}%
\pgfpathlineto{\pgfqpoint{1.258074in}{0.667572in}}%
\pgfpathlineto{\pgfqpoint{1.545131in}{0.638273in}}%
\pgfpathlineto{\pgfqpoint{1.832189in}{0.660247in}}%
\pgfpathlineto{\pgfqpoint{2.119246in}{0.704195in}}%
\pgfpathlineto{\pgfqpoint{2.406303in}{0.652923in}}%
\pgfpathlineto{\pgfqpoint{2.693361in}{0.682221in}}%
\pgfpathlineto{\pgfqpoint{2.980418in}{0.667572in}}%
\pgfusepath{stroke}%
\end{pgfscope}%
\begin{pgfscope}%
\pgfpathrectangle{\pgfqpoint{0.569136in}{0.499691in}}{\pgfqpoint{2.526105in}{3.048807in}}%
\pgfusepath{clip}%
\pgfsetbuttcap%
\pgfsetroundjoin%
\pgfsetlinewidth{1.505625pt}%
\definecolor{currentstroke}{rgb}{0.000000,0.000000,0.000000}%
\pgfsetstrokecolor{currentstroke}%
\pgfsetstrokeopacity{0.500000}%
\pgfsetdash{{5.550000pt}{2.400000pt}}{0.000000pt}%
\pgfpathmoveto{\pgfqpoint{2.119246in}{0.499691in}}%
\pgfpathlineto{\pgfqpoint{2.119246in}{3.548498in}}%
\pgfusepath{stroke}%
\end{pgfscope}%
\begin{pgfscope}%
\pgfsetrectcap%
\pgfsetmiterjoin%
\pgfsetlinewidth{0.803000pt}%
\definecolor{currentstroke}{rgb}{0.000000,0.000000,0.000000}%
\pgfsetstrokecolor{currentstroke}%
\pgfsetdash{}{0pt}%
\pgfpathmoveto{\pgfqpoint{0.569136in}{0.499691in}}%
\pgfpathlineto{\pgfqpoint{0.569136in}{3.548498in}}%
\pgfusepath{stroke}%
\end{pgfscope}%
\begin{pgfscope}%
\pgfsetrectcap%
\pgfsetmiterjoin%
\pgfsetlinewidth{0.803000pt}%
\definecolor{currentstroke}{rgb}{0.000000,0.000000,0.000000}%
\pgfsetstrokecolor{currentstroke}%
\pgfsetdash{}{0pt}%
\pgfpathmoveto{\pgfqpoint{3.095241in}{0.499691in}}%
\pgfpathlineto{\pgfqpoint{3.095241in}{3.548498in}}%
\pgfusepath{stroke}%
\end{pgfscope}%
\begin{pgfscope}%
\pgfsetrectcap%
\pgfsetmiterjoin%
\pgfsetlinewidth{0.803000pt}%
\definecolor{currentstroke}{rgb}{0.000000,0.000000,0.000000}%
\pgfsetstrokecolor{currentstroke}%
\pgfsetdash{}{0pt}%
\pgfpathmoveto{\pgfqpoint{0.569136in}{0.499691in}}%
\pgfpathlineto{\pgfqpoint{3.095241in}{0.499691in}}%
\pgfusepath{stroke}%
\end{pgfscope}%
\begin{pgfscope}%
\pgfsetrectcap%
\pgfsetmiterjoin%
\pgfsetlinewidth{0.803000pt}%
\definecolor{currentstroke}{rgb}{0.000000,0.000000,0.000000}%
\pgfsetstrokecolor{currentstroke}%
\pgfsetdash{}{0pt}%
\pgfpathmoveto{\pgfqpoint{0.569136in}{3.548498in}}%
\pgfpathlineto{\pgfqpoint{3.095241in}{3.548498in}}%
\pgfusepath{stroke}%
\end{pgfscope}%
\begin{pgfscope}%
\pgfsetbuttcap%
\pgfsetmiterjoin%
\definecolor{currentfill}{rgb}{1.000000,1.000000,1.000000}%
\pgfsetfillcolor{currentfill}%
\pgfsetfillopacity{0.800000}%
\pgfsetlinewidth{1.003750pt}%
\definecolor{currentstroke}{rgb}{0.800000,0.800000,0.800000}%
\pgfsetstrokecolor{currentstroke}%
\pgfsetstrokeopacity{0.800000}%
\pgfsetdash{}{0pt}%
\pgfpathmoveto{\pgfqpoint{0.588581in}{3.609475in}}%
\pgfpathlineto{\pgfqpoint{3.075797in}{3.609475in}}%
\pgfpathquadraticcurveto{\pgfqpoint{3.095241in}{3.609475in}}{\pgfqpoint{3.095241in}{3.628919in}}%
\pgfpathlineto{\pgfqpoint{3.095241in}{5.077530in}}%
\pgfpathquadraticcurveto{\pgfqpoint{3.095241in}{5.096975in}}{\pgfqpoint{3.075797in}{5.096975in}}%
\pgfpathlineto{\pgfqpoint{0.588581in}{5.096975in}}%
\pgfpathquadraticcurveto{\pgfqpoint{0.569136in}{5.096975in}}{\pgfqpoint{0.569136in}{5.077530in}}%
\pgfpathlineto{\pgfqpoint{0.569136in}{3.628919in}}%
\pgfpathquadraticcurveto{\pgfqpoint{0.569136in}{3.609475in}}{\pgfqpoint{0.588581in}{3.609475in}}%
\pgfpathlineto{\pgfqpoint{0.588581in}{3.609475in}}%
\pgfpathclose%
\pgfusepath{stroke,fill}%
\end{pgfscope}%
\begin{pgfscope}%
\pgfsetrectcap%
\pgfsetroundjoin%
\pgfsetlinewidth{1.505625pt}%
\definecolor{currentstroke}{rgb}{0.090196,0.745098,0.811765}%
\pgfsetstrokecolor{currentstroke}%
\pgfsetdash{}{0pt}%
\pgfpathmoveto{\pgfqpoint{0.608025in}{5.019197in}}%
\pgfpathlineto{\pgfqpoint{0.705248in}{5.019197in}}%
\pgfpathlineto{\pgfqpoint{0.802470in}{5.019197in}}%
\pgfusepath{stroke}%
\end{pgfscope}%
\begin{pgfscope}%
\definecolor{textcolor}{rgb}{0.000000,0.000000,0.000000}%
\pgfsetstrokecolor{textcolor}%
\pgfsetfillcolor{textcolor}%
\pgftext[x=0.880248in,y=4.985169in,left,base]{\color{textcolor}\rmfamily\fontsize{7.000000}{8.400000}\selectfont od --format=c /dev/urandom}%
\end{pgfscope}%
\begin{pgfscope}%
\pgfsetrectcap%
\pgfsetroundjoin%
\pgfsetlinewidth{1.505625pt}%
\definecolor{currentstroke}{rgb}{0.839216,0.152941,0.156863}%
\pgfsetstrokecolor{currentstroke}%
\pgfsetdash{}{0pt}%
\pgfpathmoveto{\pgfqpoint{0.608025in}{4.873363in}}%
\pgfpathlineto{\pgfqpoint{0.705248in}{4.873363in}}%
\pgfpathlineto{\pgfqpoint{0.802470in}{4.873363in}}%
\pgfusepath{stroke}%
\end{pgfscope}%
\begin{pgfscope}%
\definecolor{textcolor}{rgb}{0.000000,0.000000,0.000000}%
\pgfsetstrokecolor{textcolor}%
\pgfsetfillcolor{textcolor}%
\pgftext[x=0.880248in,y=4.839336in,left,base]{\color{textcolor}\rmfamily\fontsize{7.000000}{8.400000}\selectfont od --format=x /dev/urandom}%
\end{pgfscope}%
\begin{pgfscope}%
\pgfsetrectcap%
\pgfsetroundjoin%
\pgfsetlinewidth{1.505625pt}%
\definecolor{currentstroke}{rgb}{0.737255,0.741176,0.133333}%
\pgfsetstrokecolor{currentstroke}%
\pgfsetdash{}{0pt}%
\pgfpathmoveto{\pgfqpoint{0.608025in}{4.727530in}}%
\pgfpathlineto{\pgfqpoint{0.705248in}{4.727530in}}%
\pgfpathlineto{\pgfqpoint{0.802470in}{4.727530in}}%
\pgfusepath{stroke}%
\end{pgfscope}%
\begin{pgfscope}%
\definecolor{textcolor}{rgb}{0.000000,0.000000,0.000000}%
\pgfsetstrokecolor{textcolor}%
\pgfsetfillcolor{textcolor}%
\pgftext[x=0.880248in,y=4.693502in,left,base]{\color{textcolor}\rmfamily\fontsize{7.000000}{8.400000}\selectfont od --format=a /dev/urandom}%
\end{pgfscope}%
\begin{pgfscope}%
\pgfsetrectcap%
\pgfsetroundjoin%
\pgfsetlinewidth{1.505625pt}%
\definecolor{currentstroke}{rgb}{0.890196,0.466667,0.760784}%
\pgfsetstrokecolor{currentstroke}%
\pgfsetdash{}{0pt}%
\pgfpathmoveto{\pgfqpoint{0.608025in}{4.581697in}}%
\pgfpathlineto{\pgfqpoint{0.705248in}{4.581697in}}%
\pgfpathlineto{\pgfqpoint{0.802470in}{4.581697in}}%
\pgfusepath{stroke}%
\end{pgfscope}%
\begin{pgfscope}%
\definecolor{textcolor}{rgb}{0.000000,0.000000,0.000000}%
\pgfsetstrokecolor{textcolor}%
\pgfsetfillcolor{textcolor}%
\pgftext[x=0.880248in,y=4.547669in,left,base]{\color{textcolor}\rmfamily\fontsize{7.000000}{8.400000}\selectfont od --format=d1 /dev/urandom}%
\end{pgfscope}%
\begin{pgfscope}%
\pgfsetbuttcap%
\pgfsetroundjoin%
\pgfsetlinewidth{1.505625pt}%
\definecolor{currentstroke}{rgb}{0.121569,0.466667,0.705882}%
\pgfsetstrokecolor{currentstroke}%
\pgfsetdash{{9.600000pt}{2.400000pt}{1.500000pt}{2.400000pt}}{0.000000pt}%
\pgfpathmoveto{\pgfqpoint{0.608025in}{4.435863in}}%
\pgfpathlineto{\pgfqpoint{0.705248in}{4.435863in}}%
\pgfpathlineto{\pgfqpoint{0.802470in}{4.435863in}}%
\pgfusepath{stroke}%
\end{pgfscope}%
\begin{pgfscope}%
\definecolor{textcolor}{rgb}{0.000000,0.000000,0.000000}%
\pgfsetstrokecolor{textcolor}%
\pgfsetfillcolor{textcolor}%
\pgftext[x=0.880248in,y=4.401836in,left,base]{\color{textcolor}\rmfamily\fontsize{7.000000}{8.400000}\selectfont od --format=f /dev/urandom}%
\end{pgfscope}%
\begin{pgfscope}%
\pgfsetrectcap%
\pgfsetroundjoin%
\pgfsetlinewidth{1.505625pt}%
\definecolor{currentstroke}{rgb}{0.172549,0.627451,0.172549}%
\pgfsetstrokecolor{currentstroke}%
\pgfsetdash{}{0pt}%
\pgfpathmoveto{\pgfqpoint{0.608025in}{4.290030in}}%
\pgfpathlineto{\pgfqpoint{0.705248in}{4.290030in}}%
\pgfpathlineto{\pgfqpoint{0.802470in}{4.290030in}}%
\pgfusepath{stroke}%
\end{pgfscope}%
\begin{pgfscope}%
\definecolor{textcolor}{rgb}{0.000000,0.000000,0.000000}%
\pgfsetstrokecolor{textcolor}%
\pgfsetfillcolor{textcolor}%
\pgftext[x=0.880248in,y=4.256002in,left,base]{\color{textcolor}\rmfamily\fontsize{7.000000}{8.400000}\selectfont base64 /dev/urandom | head -c 1G}%
\end{pgfscope}%
\begin{pgfscope}%
\pgfsetrectcap%
\pgfsetroundjoin%
\pgfsetlinewidth{1.505625pt}%
\definecolor{currentstroke}{rgb}{0.580392,0.403922,0.741176}%
\pgfsetstrokecolor{currentstroke}%
\pgfsetdash{}{0pt}%
\pgfpathmoveto{\pgfqpoint{0.608025in}{4.144197in}}%
\pgfpathlineto{\pgfqpoint{0.705248in}{4.144197in}}%
\pgfpathlineto{\pgfqpoint{0.802470in}{4.144197in}}%
\pgfusepath{stroke}%
\end{pgfscope}%
\begin{pgfscope}%
\definecolor{textcolor}{rgb}{0.000000,0.000000,0.000000}%
\pgfsetstrokecolor{textcolor}%
\pgfsetfillcolor{textcolor}%
\pgftext[x=0.880248in,y=4.110169in,left,base]{\color{textcolor}\rmfamily\fontsize{7.000000}{8.400000}\selectfont cat /dev/urandom | tr -dc 'a-zA-Z0-9' | ...}%
\end{pgfscope}%
\begin{pgfscope}%
\pgfsetrectcap%
\pgfsetroundjoin%
\pgfsetlinewidth{1.505625pt}%
\definecolor{currentstroke}{rgb}{1.000000,0.498039,0.054902}%
\pgfsetstrokecolor{currentstroke}%
\pgfsetdash{}{0pt}%
\pgfpathmoveto{\pgfqpoint{0.608025in}{3.998363in}}%
\pgfpathlineto{\pgfqpoint{0.705248in}{3.998363in}}%
\pgfpathlineto{\pgfqpoint{0.802470in}{3.998363in}}%
\pgfusepath{stroke}%
\end{pgfscope}%
\begin{pgfscope}%
\definecolor{textcolor}{rgb}{0.000000,0.000000,0.000000}%
\pgfsetstrokecolor{textcolor}%
\pgfsetfillcolor{textcolor}%
\pgftext[x=0.880248in,y=3.964336in,left,base]{\color{textcolor}\rmfamily\fontsize{7.000000}{8.400000}\selectfont head -c 1G /dev/urandom}%
\end{pgfscope}%
\begin{pgfscope}%
\pgfsetrectcap%
\pgfsetroundjoin%
\pgfsetlinewidth{1.505625pt}%
\definecolor{currentstroke}{rgb}{0.549020,0.337255,0.294118}%
\pgfsetstrokecolor{currentstroke}%
\pgfsetdash{}{0pt}%
\pgfpathmoveto{\pgfqpoint{0.608025in}{3.852530in}}%
\pgfpathlineto{\pgfqpoint{0.705248in}{3.852530in}}%
\pgfpathlineto{\pgfqpoint{0.802470in}{3.852530in}}%
\pgfusepath{stroke}%
\end{pgfscope}%
\begin{pgfscope}%
\definecolor{textcolor}{rgb}{0.000000,0.000000,0.000000}%
\pgfsetstrokecolor{textcolor}%
\pgfsetfillcolor{textcolor}%
\pgftext[x=0.880248in,y=3.818502in,left,base]{\color{textcolor}\rmfamily\fontsize{7.000000}{8.400000}\selectfont openssl rand -out myfile \$(echo 1G | nu...}%
\end{pgfscope}%
\begin{pgfscope}%
\pgfsetrectcap%
\pgfsetroundjoin%
\pgfsetlinewidth{1.505625pt}%
\definecolor{currentstroke}{rgb}{0.121569,0.466667,0.705882}%
\pgfsetstrokecolor{currentstroke}%
\pgfsetdash{}{0pt}%
\pgfpathmoveto{\pgfqpoint{0.608025in}{3.706697in}}%
\pgfpathlineto{\pgfqpoint{0.705248in}{3.706697in}}%
\pgfpathlineto{\pgfqpoint{0.802470in}{3.706697in}}%
\pgfusepath{stroke}%
\end{pgfscope}%
\begin{pgfscope}%
\definecolor{textcolor}{rgb}{0.000000,0.000000,0.000000}%
\pgfsetstrokecolor{textcolor}%
\pgfsetfillcolor{textcolor}%
\pgftext[x=0.880248in,y=3.672669in,left,base]{\color{textcolor}\rmfamily\fontsize{7.000000}{8.400000}\selectfont ./lorem/lorem -c 1000000000}%
\end{pgfscope}%
\end{pgfpicture}%
\makeatother%
\endgroup%

  \end{center}
  \caption{Decompression times for files generated with ~od~'s best output formats and the other tools.}
  \label{fig:all}
\end{figure}

\subsection{Other results}
\label{sec:orgb1809d0}

To verify that our results were not specific to our machine, we also ran the
same tests on a MacBook Pro (late 2020) with an M1 CPU, 16GB of RAM, and
running macOS Ventura 13.3.1 (a). We used the same version of \texttt{gzip} as on
Ubuntu (1.12). Results were similar, although the MacBook Pro was on average
faster than the Ubuntu machine. We thus do not include the results in this
report.

Furthermore, we tried various file sizes, between 100MB and 1GB, in order to
verify there was also no correlation between the file size and the results.
Again, we found that the results were comparable, and thus we only reported
the results once for the 100MB files.


\section{Conclusions}
\label{sec:orgd43c62f}

We have shown that the decompression time of a file can vary significantly
depending on the content of the file and the compression level used. Indeed,
we have shown that the decompression time of a file can be increased by a
factor of 3 when files are filled with datafrom \texttt{od}'s \texttt{a} and \texttt{c} output formats,
when compared to the decompression time of a file containing English text.

We believe that this is a serious issue, as it can be used to artificially
increase the decompression time of a file, which can be used to slow down
systems that rely on gzip for decompression. For example, this could be used
to slow down the unpacking of Docker images, which use gzip for compression
of the various layers.

\section{Bibliography}
\label{sec:org69fa9ca}

\bibliographystyle{plain}
\bibliography{/Users/matte/Codice/fbk/gziptests/report/gzip}

\section{Appendix}
\label{sec:org6bcffaf}

The following are the full results of our tests for the last graph in the
previous section. 

%\begin{xtabular}{|p{1cm}p{1.2cm}p{1.2cm}p{1.2cm}p{1.5cm}|}
    \hline
    Level & \makecell[l]{Compr.\\ratio} & \makecell[l]{Compr.\\size} &  \makecell[l]{Compr.\\time} & \makecell[l]{Decompr.\\time} \\
    \hline
\multicolumn{5}{|c|}{od --format=c ...} \\
1 & 0.540 & 58M & 1.720 & 0.906 \\
2 & 0.531 & 57M & 1.950 & 0.914 \\
3 & 0.522 & 56M & 3.120 & 0.862 \\
4 & 0.513 & 55M & 2.860 & 0.860 \\
5 & 0.513 & 55M & 5.270 & 0.848 \\
6 & 0.504 & 54M & 9.230 & 0.826 \\
7 & 0.504 & 54M & 9.390 & 0.828 \\
8 & 0.504 & 54M & 9.420 & 0.820 \\
9 & 0.504 & 54M & 9.440 & 0.830 \\
\hline
\multicolumn{5}{|c|}{od --format=x ...} \\
1 & 0.590 & 56M & 1.170 & 0.754 \\
2 & 0.590 & 56M & 1.550 & 0.780 \\
3 & 0.580 & 55M & 1.870 & 0.772 \\
4 & 0.580 & 55M & 2.500 & 0.826 \\
5 & 0.580 & 55M & 3.040 & 0.830 \\
6 & 0.580 & 55M & 3.050 & 0.814 \\
7 & 0.580 & 55M & 3.020 & 0.810 \\
8 & 0.580 & 55M & 3.040 & 0.816 \\
9 & 0.580 & 55M & 3.100 & 0.858 \\
\hline
\multicolumn{5}{|c|}{od --format=a ...} \\
1 & 0.717 & 77M & 2.240 & 0.888 \\
2 & 0.717 & 77M & 2.430 & 0.902 \\
3 & 0.717 & 77M & 2.840 & 0.910 \\
4 & 0.717 & 77M & 3.050 & 0.824 \\
5 & 0.717 & 77M & 4.650 & 0.820 \\
6 & 0.717 & 77M & 5.500 & 0.812 \\
7 & 0.717 & 77M & 6.430 & 0.818 \\
8 & 0.717 & 77M & 6.340 & 0.808 \\
9 & 0.717 & 77M & 6.330 & 0.800 \\
\hline
\multicolumn{5}{|c|}{od --format=d1...} \\
1 & 0.442 & 48M & 1.240 & 0.834 \\
2 & 0.442 & 48M & 1.480 & 0.822 \\
3 & 0.425 & 46M & 2.800 & 0.744 \\
4 & 0.425 & 46M & 2.230 & 0.752 \\
5 & 0.407 & 44M & 5.000 & 0.758 \\
6 & 0.398 & 43M & 8.480 & 0.752 \\
7 & 0.407 & 44M & 9.590 & 0.750 \\
8 & 0.398 & 43M & 11.000 & 0.774 \\
9 & 0.407 & 44M & 11.120 & 0.768 \\
\hline
\multicolumn{5}{|c|}{od --format=f ...} \\
1 & 0.469 & 51M & 1.240 & 0.828 \\
2 & 0.460 & 50M & 1.480 & 0.822 \\
3 & 0.442 & 48M & 2.490 & 0.714 \\
4 & 0.442 & 48M & 2.200 & 0.718 \\
5 & 0.434 & 47M & 4.350 & 0.738 \\
6 & 0.434 & 47M & 5.530 & 0.730 \\
7 & 0.434 & 47M & 6.340 & 0.758 \\
8 & 0.434 & 47M & 6.280 & 0.752 \\
9 & 0.434 & 47M & 6.120 & 0.736 \\
\hline
\multicolumn{5}{|c|}{base64 /dev/ur...} \\
1 & 0.810 & 77M & 3.220 & 0.790 \\
2 & 0.810 & 77M & 3.230 & 0.798 \\
3 & 0.810 & 77M & 3.200 & 0.788 \\
4 & 0.810 & 77M & 3.670 & 0.698 \\
5 & 0.810 & 77M & 3.670 & 0.696 \\
6 & 0.810 & 77M & 3.670 & 0.688 \\
7 & 0.810 & 77M & 3.660 & 0.690 \\
8 & 0.810 & 77M & 3.630 & 0.688 \\
9 & 0.810 & 77M & 3.730 & 0.692 \\
\hline
\multicolumn{5}{|c|}{cat /dev/urand...} \\
1 & 0.717 & 77M & 3.040 & 0.776 \\
2 & 0.717 & 77M & 3.150 & 0.776 \\
3 & 0.717 & 77M & 3.140 & 0.778 \\
4 & 0.717 & 77M & 3.620 & 0.670 \\
5 & 0.717 & 77M & 3.640 & 0.672 \\
6 & 0.717 & 77M & 3.640 & 0.670 \\
7 & 0.717 & 77M & 3.640 & 0.670 \\
8 & 0.717 & 77M & 3.630 & 0.676 \\
9 & 0.717 & 77M & 3.630 & 0.672 \\
\hline
\multicolumn{5}{|c|}{head -c 1G /de...} \\
1 & 1.130 & 108M & 2.100 & 0.430 \\
2 & 1.130 & 108M & 2.110 & 0.432 \\
3 & 1.130 & 108M & 2.130 & 0.430 \\
4 & 1.130 & 108M & 2.300 & 0.450 \\
5 & 1.130 & 108M & 2.300 & 0.432 \\
6 & 1.130 & 108M & 2.260 & 0.436 \\
7 & 1.130 & 108M & 2.320 & 0.444 \\
8 & 1.130 & 108M & 2.270 & 0.436 \\
9 & 1.130 & 108M & 2.250 & 0.436 \\
\hline
\multicolumn{5}{|c|}{openssl rand -...} \\
1 & 1.130 & 108M & 2.110 & 0.420 \\
2 & 1.130 & 108M & 2.100 & 0.420 \\
3 & 1.130 & 108M & 2.110 & 0.424 \\
4 & 1.130 & 108M & 2.220 & 0.422 \\
5 & 1.130 & 108M & 2.200 & 0.424 \\
6 & 1.130 & 108M & 2.210 & 0.426 \\
7 & 1.130 & 108M & 2.210 & 0.434 \\
8 & 1.130 & 108M & 2.220 & 0.430 \\
9 & 1.130 & 108M & 2.650 & 0.436 \\
\hline
\multicolumn{5}{|c|}{./lorem/lorem ...} \\
1 & 0.014 & 1M & 0.330 & 0.292 \\
2 & 0.014 & 1M & 0.320 & 0.293 \\
3 & 0.009 & 801k & 0.300 & 0.290 \\
4 & 0.005 & 465k & 0.390 & 0.283 \\
5 & 0.005 & 465k & 0.390 & 0.288 \\
6 & 0.005 & 465k & 0.380 & 0.298 \\
7 & 0.005 & 465k & 0.380 & 0.287 \\
8 & 0.005 & 465k & 0.380 & 0.293 \\
9 & 0.005 & 465k & 0.380 & 0.290 \\
\hline
\end{xtabular}
\end{document}